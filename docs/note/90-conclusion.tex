%% ГОСТ 7.32-2017
%% 5.9 Заключение
%%
%% Заключение должно содержать:
%% - краткие выводы по результатам выполненной НИР или отдельных её этапов;
%% - оценку полноты решений поставленных задач;
%% - разработку рекомендаций и исходных данных по конкретному использованию результатов НИР;
%% - результаты оценки технико-экономической эффективности внедрения;
%% - результаты оценки научно-технического уровня выполненной НИР в сравнении с лучшими достижениями в этой области.

%% Методические указания к выполнению, оформлению и защите выпускной квалификационной работы бакалавра
%% 2.10 Заключение
%%
%% Заключение содержит краткие выводы по всей работе и оценку полноты решения поставленной задачи.

\StructuralElement{Заключение}

На основе разработанной математической модели проведено исследование квазистационарных режимов мощных разрядов в труб-чатых ксеноновых источниках излучения в условиях, когда они нахо-дятся под воздействием собственного отраженного излучения. Полу-чены данные по влиянию возвращаемой энергии на температурные по-ля в плазме. Показано, что свойства плазмы могут заметно меняться в зависимости от значений спектральных коэффициентов возврата и оптической плотности среды.

По итогам работы получены следующие результаты:

\begin{enumerate}
	\item Разработана дискретно-лучевая модель прохождения излучения в системе сред и поверхностей, связанных единым электро-магнитным полем, формируемым разрядными источниками излучения и отражателями.
	Источники излучения имеют неоднородное по объёму плазмы распределение оптических свойств и температуры, что требует детального рассмотрения  физики  взаимодействия луча с плазменной средой.

	\item Разработан метод расчёта генерации излучения из объёма неоднородной среды в условиях цилиндрической симметрии, пригодный для моделирования начального веса луча при произвольной оптической плотности плазмы.
	Выполнено сравнение двух подходов к выбору стартовой позиции луча: из внутренних точек объёма и с поверхности цилиндрического плазменного столба.
	Показано, что второй способ обеспечивает почти полуторакратное превышение по быстродействию.

	\item Разработан метод расчёта распределения поглощенной мощности излучения по объёму плазменного цилиндра, в том числе и в условиях засветки его соседними источниками излучения в многоламповых системах.
	Тем самым созданы предпосылки для построения замкнутых  моделей развития процессов в плазме, находящейся  в составе оптической системы и подсвечиваемой внешним излучением. Показано, что распределение может иметь существенную радиальную и азимутальную неоднородность, что заметно искажает темпера-турные и концентрационные поля в разряде по сравнению с открытым излучателем и может отразиться на результатах вычислений теплофизических и спектрально-энергетических характеристик разрядов.

	\item Разработаны алгоритмы, реализующие указанные методы.
	Выполнено распараллеливание для многопроцессорных вычислителей на основе технологии CUDA.
	Показано, что заметный эффект от перехода к параллельной архитектуре ПМО может быть достигнут при существенной оптимизации на этапе получения распределения интегральных по спектру удельных источников тепловыделения.

	\item Разработан оригинальный программно-алгоритмический комплекс с интерфейсом пользователя, удобным для автоматизации научных исследований.
	ПМО построено на платформе…

	\item Проведены исследования методов и реализующих их алгоритмов, выполнен сравнительный анализ различных вариантов методов, показаны преимущества и недостатки разных подходов и реализаций.
	Проведено сравнение результатов распределения поглощенной мощности в плазме, полученных по методу настоящей работы и в дифференциальном приближении.
	Показано, что при определенных параметрах последнее может давать ошибку до 12-17~\%.

	\item Выполнены исследования в предметной области.
	Рассмотрены два варианта систем: система накачки лазера на алюмо-иттриевом гранате с моноблоком и система разряд-оболочка-интерференционное покрытие.
	Показано:
	\begin{enumerate}
		\item
		\item
		\item
		\item
		\item
	\end{enumerate}
\end{enumerate}

\Accent{Дальнейшее развитие работы} предполагает создание  метода и программного модуля трассировки лучей в системах, использующих светорассеивающие среды типа поликора или жидких активных элементов.
Также значительный эффект с точки зрения увеличения быстродействия может дать оптимизация многолучевой трассировки при массовых расчётах однотипных систем с отличающимися параметрами разряда.
