%% ГОСТ 7.32-2017
%% 5.7 Введение
%%
%% 5.7.1 Введение должно содержать оценку современного состояния решаемой научно-технической проблемы, основание и исходные данные для разработки темы, обоснование необходимости проведения НИР, сведения о планируемом научно-техническом уровне разработки, о патентных исследованиях и выводы из них, сведения о метрологическом обеспечении НИР.
%% Во введении должны быть отражены актуальность и новизна темы, связь данной работы с другими научно-исследовательскими работами.
%%
%% 5.7.2 Во введении промежуточного отчета по этапу НИР должны быть указаны цели и задачи исследований, выполненных на данном этапе, их место в выполнении отчета о НИР в целом.
%%
%% 5.7.3 Во введении заключительного отчета о НИР приводят перечень наименований всех подготовленных промежуточных отчетов по этапам и их регистрационные номера, если они были представлены в соответствующий орган [1] для регистрации.
%%   [1] В Российской Федерации — ЦИТиС, который присваивает эти номера при представлении промежуточного отчета на регистрацию.


%% Методические указания к выполнению, оформлению и защите выпускной квалификационной работы бакалавра
%% 2.3 Введение
%%
%% Во введении обосновывается актуальность выбранной темы (со ссылками на монографии, научные статьи), формулируется цель работы («Целью работы является...») и перечисляются задачи, которые необходимо решить для достижения этой цели («Для достижения поставленной цели необходимо решить следующие задачи...»)
%%
%% Среди задач, как правило, выделяют аналитические, конструкторские, технологические и исследовательские.
%% Решение этих задач описывается в соответствующих разделах.
%%
%% Рекомендуемый объём введения 2—3 страницы.


\StructuralElement{Введение}

\Accent{Объектом исследования} являются математические модели систем с разрядными источниками мощного селективного излучения и реализующие эти модели программно-алгоритмические средства.
Указанные системы могут быть идентифицированы как системы, назначением и основой функционирования которых является интенсивное радиационное воздействие на материалы, среды и процессы.
Речь идёт о системах накачки лазеров, различного типа облучательных установках, технологических процессах, базирующихся на фотохимическом и фотобиологическом действиях света, светотехнических устройствах самого широкого назначения и~т.~д. \cite{article-1-marshak, article-2-hancock, article-3-moreau}.

\Accent{Предмет исследования} — методы и параллельные алгоритмы расчёта распространения излучения дискретно-непрерывного спектрального состава в сложной системе поглощающе-излучающих сред и поверхностей с учётом их реальных оптических свойств и неоднородности распределения этих свойств по объёму элементов.

\Accent{Актуальность работы}

В работе решается задача моделирования процессов в плазменном источнике селективного излучения, находящемся в составе некоторой осветительной системы, частично возвращающей излучение в плазму.
Эта задача может рассматриваться как часть более общей проблемы моделирования систем, содержащих мощный источник излучения, элементы которых связаны единым радиационным полем.
Поле генерируется разрядом в различных плазмообразующих средах и формируется многочисленными средами и поверхностями, предназначенными для эффективной передачи энергии приемнику.
Проектирование, оптимизация, правильная эксплуатация столь сложных систем, понимание перспективности различных направлений научно-технического развития неизбежно выдвигает на передний план моделирование как основную методологическую базу научных исследований.
Вычислительный эксперимент становится решающим способом получения новых данных о функционировании систем и прогнозирования их характеристик ещё до создания натурных образцов изделий.

Соответственно, в ведущих научных организациях нашей страны и за рубежом (Институт проблем механики им. А.~Ю.~Ишлинского РАН, Институт прикладной математики им. М.~В.~Келдыша РАН, НПО «ГОИ им. С.~И.~Вавилова, Ливерморская национальная лаборатория им. Э.~Лоуренса» и др.) разработаны и успешно эксплуатируются математические модели и реализующие их пакеты программ, позволяющие с той или иной степенью достоверности получать нужные характеристики обсуждаемых систем.
Эти модели ориентированы на расчёты в основном систем накачки твердотельных лазеров, в первую очередь применяемых в установках лазерного термоядерного синтеза класса Mega Science.

В этих системах при массовых расчётах для плазмы источников накачки применяется модель изотермической однородной среды, а основное внимание сосредоточено на правильном учёте прохождения излучения в системе и поглощения его в активной среде.
В то же время появление новых плазмообразующих сред и баз данных по оптическим свойствам для них, необходимость для ряда задач «тонкой» проработки самообращённых резонансных линий, проблематика повышения эффективности излучения широкополосных источников света в узких интервалах спектра за счёт нанесения на оболочку интерференционных покрытий остро ставит вопрос о более детальном моделировании процессов излучения и прохождения квантов через плазменные слои, имеющие сильно неоднородные по пространству разряда температурные поля и поля концентраций частиц.
При этом во многом создание и реализация более совершенных моделей и методов, если учесть ещё и сложный характер спектра излучения дискретно-непрерывного состава с большим числом компонент, упирается в алгоритмическую сложность их реализации.
В этой ситуации весьма актуальной становится задача создания моделей, методов, алгоритмов и программ, позволяющих реалистично рассматривать взаимодействие плазменной среды с собственным радиационным полем сложного спектрального состава в условиях её существенной неоднородности.
Массовость же применения методологии моделирования в задачах реального проектирования и тем более  оптимизации новых образцов техники выдвигает достаточно жесткие требования к необходимым вычислительным ресурсам, что стимулирует интерес к разработке программно-математического обеспечения с параллельной архитектурой, доступного для использования на серийной компьютерной базе.

\Accent{Целью} выпускной квалификационной работы является создание дискретно-лучевого метода и программно-алгоритмического комплекса с параллельной архитектурой для моделирования распространения радиации сложного дискретно-непрерывного состава в оптических системах с неоднородными поглощающими и излучающими средами.

%TODO(a.kerimov)
\Accent{Задачи:}

\begin{enumerate}
	\item провести анализ существующих методов моделирования осветительных систем;
	\item разработать дискретно-лучевой метод моделирования световых полей в неоднородных средах на основе параллельных вычислений;
	\item программно реализовать разработанный метод;
	\item провести исследования метода и модели, широкомасштабные численные эксперименты в предметной области.
\end{enumerate}

%\begin{enumerate}
	%\item Критический анализ литературных данных по проблеме моделирования «тесных» осветительных систем с объёмными источниками излучения сложного спектрального состава плазменного типа с сильной пространственной неоднородностью.
	%Изучение возможности использования существующих программных средств для решения задачи.

	%\item Построение метода трассировки лучей в осветительной системе заданной конфигурации, содержащей неоднородные по температуре и коэффициенту оптического поглощения объёмные источники излучения с расчётом распределения поглощенной в плазме мощности.

	%\item Разработка параллельных вычислительных алгоритмов, реализующих метод, программных и пользовательских интерфейсов.

	%\item Разработка и тестирование программного комплекса.

	%\item Исследование влияния различных факторов на затраты вычислительных ресурсов, эффективности различных способов распараллеливания и оптимизации вычислений, проведение широкомасштабных численных экспериментов, подтверждающих достоверность получаемых результатов, и выработка рекомендаций по техническим решениям в рассматриваемой предметной области.
%\end{enumerate}
