%% Методические указания к выполнению, оформлению и защите выпускной квалификационной работы бакалавра
%% 2.4 Аналитический раздел
%%
%% В данном разделе расчетно-пояснительной записки проводится анализ предметной области и выделяется основной объект исследования.
%% Если формализовать предметную область с помощью математической модели не удается и при этом она сложна для понимания, то для отображения происходящих в ней процессов необходимо использовать методологию IDEF0, а для описания сущностей предметной области и взаимосвязей между ними — ER-модель.
%%
%% Затем выполняется обзор существующих методов и алгоритмов решения идентифицированной проблемы предметной области (опять же с обязательными ссылками на научные источники: монографии, статьи и др.) и их программных реализаций (при наличии), анализируются достоинства и недостатки каждого из них.
%% Выполненный обзор должен позволить объективно оценить актуальное состояние изучаемой проблемы.
%% Результаты проведенного анализа по возможности классифицируются и оформляются в табличной форме.
%%
%% На основе выполненного анализа обосновывается необходимость разработки нового или адаптации существующего метода или алгоритма.
%%
%% Если же целью анализа являлся отбор (на основе четко сформулированных критериев) тех методов и алгоритмов, которые наиболее эффективно решают поставленную задачу, то форма представления результата должна подтвердить обоснованность сделанного выбора, в том числе — полноту и корректность предложенных автором критериев отбора.
%%
%% Одним из основных выводов аналитического раздела должно стать формализованное описание проблемы предметной области, на решение которой будет направлен данный проект, включающее в себя:
%% — описание входных и выходных данных;
%% — указание ограничений, в рамках которых будет разработан новый, адаптирован существующий или просто реализован метод или алгоритм;
%% — описание критериев сравнения нескольких реализаций метода или алгоритма;
%% — описание способов тестирования разработанного, адаптированного или реализованного метода или алгоритма;
%% — описание функциональных требований к разрабатываемому программному обеспечению,
%% при этом в зависимости от направления работы отдельные пункты могут отсутствовать.
%%
%% Если в результате работы будет создано программное обеспечение, реализующее большое количество типичных способов взаимодействия с пользователем, необходимо каждый из этих способов описать с помощью диаграммы прецедентов [4, 5].
%%
%% Рекомендуемый объем аналитического раздела 25—30 страниц.

\chapter{Аналитический раздел}

\section{Предметная область}

\Accent{Объект исследования}

\section{Существующие методы}

\subsection{Зональный метод}

\subsection{Метод обобщённых угловых коэффициентов}

\subsection{Метод дискретных ординат}

\subsection{Дискретно-лучевой метод}

\def\wA{5mm}
\def\wB{27mm}
\def\wC{23mm}
\def\wD{23mm}
\def\wE{23mm}
\def\wF{23mm}
\def\wG{23mm}
\def\wH{23mm}
\def\wI{22mm}
\def\wJ{22mm}
\begin{FixLineStretch}
\begin{sidewaystable}
	\small
	\caption{Сравнительный анализ существующих методов моделирования световых полей}
	\label{tbl:existing-modeling-methods}
	\begin{tabular}{|p{\wA}|p{\wB}|p{\wC}|p{\wD}|p{\wE}|p{\wF}|p{\wG}|p{\wH}|p{\wI}|p{\wJ}|}
		\hline
		\TableHeader{\wA}{№} & \TableHeader{\wB}{Метод}                                                                                & \TableHeader{\wC}{Учёт излучения плазмы источника из объёма} & \TableHeader{\wD}{Учёт неоднородного распределения параметров плазмы по объему} & \TableHeader{\wE}{Нахождение распределения поглощённой мощности по объёму плазмы} & \TableHeader{\wF}{Использование параллельных алгоритмов} & \TableHeader{\wG}{Возможность применения метода для построения замкнутых систем моделирования} & \TableHeader{\wH}{Универсальность подходов для расчёта систем с плазмой произвольной оптической плотности} & \TableHeader{\wI}{Отсутствие вероятностного розыгрыша лучей} & \TableHeader{\wJ}{Доступность ПМО и ЭВМ} \\ \hline
		1                    & \everypar{\hspace*{0pt}} Зональный \cite{radiation-heat-transfer}                                       & Нет                                                          & Нет                                                                             & Нет                                                                               & Да                                                       & Нет                                                                                            & Нет                                                                                                        & Да                                                           & Нет                                      \\ \hline
		2                    & \everypar{\hspace*{0pt}} Обобщённых угловых коэффициентов \cite{encyclopedia-of-low-temperature-plasma} & Нет                                                          & Нет                                                                             & Нет                                                                               & Да                                                       & Нет                                                                                            & Нет                                                                                                        & Да                                                           & Нет                                      \\ \hline
		3                    & \everypar{\hspace*{0pt}} Дискретных ординат \cite{surzhikov}                                            & Да                                                           & Нет                                                                             & Нет                                                                               & Да                                                       & Да                                                                                             & Нет                                                                                                        & Да                                                           & Нет                                      \\ \hline
		4                    & \everypar{\hspace*{0pt}} Дискретно-лучевой \cite{gradov-dissertation}                                   & Да                                                           & Да                                                                              & Нет                                                                               & Нет                                                      & Да                                                                                             & Да                                                                                                         & Нет                                                          & Да                                       \\ \hline
	\end{tabular}
\end{sidewaystable}
\end{FixLineStretch}
\let\wJ\relax
\let\wI\relax
\let\wH\relax
\let\wG\relax
\let\wF\relax
\let\wE\relax
\let\wD\relax
\let\wC\relax
\let\wB\relax
\let\wA\relax

\section{Формализация задачи}


\let\wF\relax