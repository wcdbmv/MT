%% Методические указания к выполнению, оформлению и защите выпускной квалификационной работы бакалавра
%% 2.4 Аналитический раздел
%%
%% В данном разделе расчетно-пояснительной записки проводится анализ предметной области и выделяется основной объект исследования.
%% Если формализовать предметную область с помощью математической модели не удается и при этом она сложна для понимания, то для отображения происходящих в ней процессов необходимо использовать методологию IDEF0, а для описания сущностей предметной области и взаимосвязей между ними — ER-модель.
%%
%% Затем выполняется обзор существующих методов и алгоритмов решения идентифицированной проблемы предметной области (опять же с обязательными ссылками на научные источники: монографии, статьи и др.) и их программных реализаций (при наличии), анализируются достоинства и недостатки каждого из них.
%% Выполненный обзор должен позволить объективно оценить актуальное состояние изучаемой проблемы.
%% Результаты проведенного анализа по возможности классифицируются и оформляются в табличной форме.
%%
%% На основе выполненного анализа обосновывается необходимость разработки нового или адаптации существующего метода или алгоритма.
%%
%% Если же целью анализа являлся отбор (на основе четко сформулированных критериев) тех методов и алгоритмов, которые наиболее эффективно решают поставленную задачу, то форма представления результата должна подтвердить обоснованность сделанного выбора, в том числе — полноту и корректность предложенных автором критериев отбора.
%%
%% Одним из основных выводов аналитического раздела должно стать формализованное описание проблемы предметной области, на решение которой будет направлен данный проект, включающее в себя:
%% — описание входных и выходных данных;
%% — указание ограничений, в рамках которых будет разработан новый, адаптирован существующий или просто реализован метод или алгоритм;
%% — описание критериев сравнения нескольких реализаций метода или алгоритма;
%% — описание способов тестирования разработанного, адаптированного или реализованного метода или алгоритма;
%% — описание функциональных требований к разрабатываемому программному обеспечению,
%% при этом в зависимости от направления работы отдельные пункты могут отсутствовать.
%%
%% Если в результате работы будет создано программное обеспечение, реализующее большое количество типичных способов взаимодействия с пользователем, необходимо каждый из этих способов описать с помощью диаграммы прецедентов [4, 5].
%%
%% Рекомендуемый объем аналитического раздела 25—30 страниц.

\chapter{Аналитический раздел}

\section{Предметная область}

В работе рассматриваются технические устройства, состоящие из некоторого количества источников и приёмников излучения.
Примерами таких устройств (рис. \ref{img:lamp-1}—\ref{img:lamp-2-3}) служит различная осветительная, сигнальная техника, системы некогерентной оптической накачки, разнообразные излучательные имитаторы, волоконно-оптические нагревательные печи, фотонные импульсные генераторы и облучательные установки \cite{lighting-engineering, lasers, neodymium-glass-lasers, sarychev}.
% NOEDITEDNEXTLINE
Последние предназначены для технологического использования фотобиологического, фотохимического, фотолюминесцентного, фотоэлектрического действий оптического излучения ультрафиолетового, видимого и инфракрасного спектральных диапазонов и отличаются особенным многообразием как по конструкции, так и по функциональному назначению.

В общем случае такие системы имеют довольно сложную конфигурацию, состоящую из:
отражателей самой разнообразной формы, различных сред, приемников и источников излучения, а также полупроводниковых пластин, лакокрасочных покрытий, активной лазерной среды и~др.
В зависимости от длины волны излучения материалы перечисленных элементов по-разному проявляют свои физико-оптические свойства.
На границах разделов сред могут происходить рассеивающее, зеркальное или смешанное отражение, преломление и поглощение.
К тому же вся система находится в едином электромагнитном поле, а источниками излучения могут выступать несколько активных элементов.

Активные излучающие источники состоят из плазменной разрядной трубки и могут быть окружены газово-жидкостным охлаждающим слоем.
В то же время оболочки трубок сами могут быть источниками электромагнитного излучения (рисунок \ref{img:lamp-2-3}).
Поглощение оптического излучения происходит селективно и зависит от материала оболочки (поликор, лейкосапфир, кварц).
В дополнении к этому само покрытие трубок может быть поглощающим или отражающим с задачей разного рода фильтрации спектральных компонент.

\img{width=0.85\linewidth}{lamp-1}{Система накачки с дисковыми активными элементами}
\imgh{width=0.9\linewidth}{lamp-2-3}{Объёмная фотохимическая установка и лампа с системой излучающих оболочек}

\imgh{width=0.85\linewidth}{krypton-efficiency}{Спектральное распределение КПД излучения разряда в криптоне. Внутренний радиус разрядной трубки $R=0,3$ см, рабочее давление в разряде $p=2,5$ МПа. \\
	а — ток $I=100$ А, средняя удельная электрическая мощность $\langle w \rangle = 8,5$ кВт/см$^3$, осевая температура плазмы $T_0=9980$ К; \\
	б — ток $I=400$ А, $\langle w \rangle = 62$ кВт/см$^3$ $T_0=11640$ К; \\
	в — ток $I=800$ А, $\langle w \rangle = 170$ кВт/cм$^3$ $T_0=12870$ К}

Режим работы излучателя определяется огромным количеством элементарных электромагнитных процессов, помимо этого зависит от конфигурации всей системы: геометрии, состава, давления и~т.~д.
Результирующие световое излучение меняется не только в зависимости от спектрального диапазона, но даже в пределах единичного импульса электрического разряда.
В качестве примера на рисунке \ref{img:krypton-efficiency} изображена зависимость кривых спектральных излучений разряда криптона от температуры в плазме, силы и мощности электрического тока.

Таким образом, \Accent{объектом исследования} являются математические модели систем с разрядными источниками мощного селективного излучения и реализующие эти модели программно-алгоритмические средства.
Указанные системы могут быть идентифицированы как системы, назначением и основой функционирования которых является интенсивное электромагнитное воздействие на материалы, среды и процессы.
Речь идет о системах накачки лазеров, различного типа облучательных установках, технологических процессах, базирующихся на фотохимическом и фотобиологическом действиях света, светотехнических устройствах самого широкого назначения и~т.~д.

\section{Существующие методы моделирования термодинамических систем}

\subsection{Зональный метод}

Согласно зональному методу, неизотермические газ и
замыкающая его оболочка разделяются на ряд объемов и площадей, которые
могут считаться близкими к изотермическим. Затем для каждой
площади и объема записывается уравнение баланса энергии. При
этом получается система уравнений относительно неизвестных
тепловых потоков или температур аналогично тому, как это было
описано в разд. 17.3 для изотермического газа. Этот метод
не является элегантным в формальном математическом смысле,
но на практике он очень полезен. Достаточно подробное описание
метода приведено Хоттелем и Сэрофим [2]. Хоттель и Коэн [15],
а также Эйнштейн [18, 19] применили его для пространственных задач.
В этом разделе рассматривается обмен энергией только
путем излучения; распространение метода на задачи, в которых
учитывается также теплопроводность и конвекция, можно найти
в гл. 19 и работе [2].

Зональный метод имеет преимущество перед методом Кертиса~—
Годсона, заключающееся в том, что этим методом можно решать
задачи с неизвестным распределением температур в газе.
Приближение Кертиса~— Годсона наиболее полезно в случае, когда
распределение температуры известно, если же оно неизвестно, то для
определения температуры газа следует применить метод итераций.

Рассмотрим основные положения зонального метода для газа,
коэффициент поглощения которого постоянен. Рассмотрим объем
$V_\gamma$ (рисунок \ref{img:ziegel-17-18}) и поверхность $A_k$. Согласно уравнению (13.33),
плотность потока излучения (без учета индуцированного
излучения) элемента объема $\mathrm dV_y$ равна $4\pi a_\lambda i'_{\lambda b}\mathrm dV_\gamma\mathrm d\lambda$, или же на единицу
телесного угла, включающего $\mathrm dV_y$ она равна $a_\lambda i'_{\lambda b}\mathrm dV_\gamma\mathrm d\lambda$. Если
смотреть со стороны элемента объема $\mathrm dV_y$, то элемент поверхности
$\mathrm dA_k$ стягивает телесный угол $\mathrm dA_k\cos{\beta_k/S^2_{\gamma-k}}$. Доля излучения,
пропускаемая на длине пути луча $S_{\gamma-k}$, равна

\begin{equation}
	\exp{\left[ -\int_{S_\gamma}^{S_k} a_\lambda \left(S^*\right) \mathrm dS^* \right]}.
\end{equation}

\img{width=0.65\linewidth}{ziegel-17-18}{Излучение от объема газа $V_\gamma$ к поверхности $A_k$}.

\noindent Перемножая эти величины и интегрируя по $V_\gamma$ и $A_k$, получим
плотность монохроматического потока излучения, падающего
на поверхность $A_k$ от объема газа $A_\gamma$,

\begin{equation}
	\mathrm dq_{\lambda i, \gamma - k}A_k = \mathrm d\lambda \int_{V_\gamma} \int_{A_k} \frac{a_\gamma(\gamma)i'_{\lambda b}(\gamma)\cos{\beta_k}}{S^2_{\gamma-k}}
	\exp{\left[ -\int_{S_\gamma}^{S_k} a_\lambda \left(S^*\right) \mathrm dS^* \right]} \mathrm dA_k \, \mathrm dV_\gamma.
\end{equation}

\noindent Если величину $a_\lambda(\gamma)$ принять постоянной, то экспоненчиальный член примет вид

\begin{equation}
	\label{eqn:ziegel-17-94}
	\exp{\left[ -a_\lambda (S_k - S_\gamma) \right]} = \tau_\lambda (S_{\gamma - k}).
\end{equation}

\noindent Полный объем газа разбивается на ряд элементарных объемов
$V_\gamma$ и предполагается, что в пределах каждого объема $V_\gamma$
параметры среды постоянны. Тогда уравнение \eqref{eqn:ziegel-17-94} упрощается
и принимает следующий вид:

\begin{equation}
	\label{eqn:ziegel-17-95}
	\mathrm dq_{\lambda i, \gamma - k}A_k = \mathrm d\lambda a_\lambda i'_{\lambda b}(\gamma)  \int_{V_\gamma} \int_{A_k} \frac{\cos{\beta_k}}{S^2_{\gamma-k}}
	\tau_\lambda (S_{\gamma - k}) \mathrm dA_k \, \mathrm dV_\gamma.
\end{equation}

\noindent Если к тому же газ еще и серый, то уравнение \eqref{eqn:ziegel-17-95}
интегрируется по всему спектру и определяется плотность интегрального
потока излучения, падающего на $A_k$:

\begin{equation}
	\label{eqn:ziegel-17-96}
	q_{i, \gamma - k}A_k = a\frac{\sigma T^4_\gamma}{\pi} \int_{V_\gamma} \int_{A_k} \frac{\cos{\beta_k}}{S^2_{\gamma-k}}
	\tau (S_{\gamma - k}) \mathrm dA_k \, \mathrm dV_\gamma.
\end{equation}

Определим теперь взаимную поверхность обмена излучением
между газом и поверхностью в виде

\begin{equation}
	\label{eqn:ziegel-17-97}
	\overline{g_\gamma s_k} \equiv \frac a\pi  \int_{V_\gamma} \int_{A_k} \frac{\cos{\beta_k}}{S^2_{\gamma-k}}
	\tau (S_{\gamma - k}) \mathrm dA_k \, \mathrm dV_\gamma.
\end{equation}

\noindent Тогда уравнение \eqref{eqn:ziegel-17-96} можно записать в виде

\begin{equation}
	q_{i, \gamma - k} A_k = \overline{g_\gamma s_k} \sigma T^4_\gamma.
\end{equation}

\noindent Следовательно, поток излучения, падающий на $A_k$, может
считаться потоком излучения черного тела $\sigma T^4_\gamma$ для газа объемом
$V_\gamma$, излучающего с эффективной поверхности $\overline{g_\gamma s_k}$.

Пусть объем газа разделен на $\Gamma$ конечных объемов. Плотность
потока излучения, падающего на элемент поверхности $A_k$ от всех
этих объемов, равна

\begin{equation}
	(q_{i, k})_{\text{от газа}} = \frac{1}{A_k} \sum_{\gamma = 1}^{\Gamma} \overline{g_\gamma s_k} \sigma T^4_\gamma.
\end{equation}

Рассмотрим теперь теплообмен излучением между граничными
поверхностями оболочки. Поток излучения, распространяющийся
от поверхности $A_j$ к поверхности $A_k$ в случае неизотермического
газа, обладающего постоянными свойствами серого газа, равен

\begin{equation}
	\label{eqn:ziegel-17-100}
	q_{i, j-k} A_k = \frac{q_{0, j}}{\pi} \int_{A_k} \int_{A_j} \tau (S_{j - k})
	\frac{\cos{\beta_j} \cos{\beta_k} \, \mathrm dA_j \, \mathrm dA_k}{S^2_{j-k}},
\end{equation}

\noindent где величина $q_{0, j}$, как в обычной теории излучения для оболочки,
считается постоянной в пределах поверхности $A_j$. Определим
теперь взаимную поверхность пары тел в виде

\begin{equation}
	\overline{s_j s_k} \equiv = \int_{A_k} \int_{A_j} \tau (S_{j - k})
	\frac{\cos{\beta_j} \cos{\beta_k} \, \mathrm dA_j \, \mathrm dA_k}{\pi S^2_{j-k}}.
\end{equation}

\noindent Тогда уравнение \eqref{eqn:ziegel-17-100} можно записать следующим образом:

\begin{equation}
	q_{i, j-k} A_k = \overline{s_j s_k} q_{0, j}.
\end{equation}

\noindent Следовательно, поток излучения $q_{i, j-k}A_k$ от $A_i$ на $A_k$ равен
произведению плотности потока эффективного излучения $q_{0, j}$
исходящего от $A_j$, на эффективную площадь $\overline{s_j s_k}$. Поток излучения,
падающий на площадку $A_k$ от всех $N$ граничных поверхностей, равен

\begin{equation}
	(q_{i, k})_{\text{от поврехн}} = \frac{1}{A_k} \sum_{j=1}^{N} \overline{s_j s_k} q_{0, j}.
\end{equation}

Тогда общий поток излучения, падающий на поверхность $A_k$,
может быть получен в виде

\begin{equation}
	\label{eqn:ziegel-17-104}
	q_{i, k} = (q_{i, k})_{\text{от поврехн}} + (q_{i, k})_{\text{от газа}} =
	\frac{1}{A_k} \left( \sum_{j=1}^{N} \overline{s_js_k}q_{0, j} + \sum_{\gamma=1}^{\Gamma} \overline{g_\gamma s_k}\sigma T^4_\gamma \right).
\end{equation}

\noindent Для поверхности $A_k$ применимы также обычные уравнения
результирующего излучения [(8.1) и (8.2)]

\begin{equation}
	q_k = q_{0, k} - q_{i, k},
\end{equation}
\begin{equation}
	\label{eqn:ziegel-17-106}
	q_{0, k} = \in_k \! \sigma T^4_k + (1 - \in_k) q_{i, k}.
\end{equation}

В тех задачах, в которых температура $T_\gamma$ задана для всех
элементарных объемов газа $V_\gamma$, уравнений \eqref{eqn:ziegel-17-104}~— \eqref{eqn:ziegel-17-106}
достаточно для решения относительно $N$ неизвестных значений
$T_k$ или $q_k$ или относительно некоторой комбинации из $N$ величин
$T_k$ и $q_k$. Другие $N$ значений $T_k$ и $q_k$ должны быть заданы в виде
граничных условий. Затем можно применить методы разд. 8.3.
Значения $\overline{s_j s_k}$ и $\overline{g_\gamma s_k}$ табулированы Хоттелем и Коэном [15] для
изотермических объемов кубической формы с изотермическими
граничными поверхностями. Хоттель и Сэрофим [2] приводят
справочную таблицу коэффициентов для одиннадцати объемов других
конфигураций и большое количество табличных данных для
объемов цилиндрической формы.

Если неизвестны температуры $T_\gamma$ для $\Gamma$ газовых объемов,
то следует найти $\Gamma$ дополнительных уравнений. Они получатся
путем составления баланса энергии для каждого элементарного
объема газа. При радиационном равновесии излучение и
поглощение в каждом элементарном объеме газа $V_\gamma$ одинаковы (в газе
нет ни тепловых стоков, ни источников). Тогда для серого газа
$c$ постоянными свойствами уравнение теплового баланса в объеме
$V_\gamma$ будет следующим:

\begin{equation}
	\label{eqn:ziegel-17-107}
	\begin{aligned}
		4a\sigma T^4_\gamma V_\gamma = & \sum_{\text{все } V_{\gamma^*}} \int_{V_\gamma} \int_{V_{\gamma^*}} \frac{4a\sigma T^4_{\gamma^*} \, \mathrm dV_{\gamma^*}}{4\pi} \tau (S_{\gamma^*-\gamma}) \frac{a \, \mathrm dV_\gamma}{S^2_{\gamma^*-\gamma}} + \\
		+ & \sum_{\text{все } A_k} \int_{V_\gamma} \int_{A_k} \frac{q_{0, k} \cos{\beta_k}}{\pi} \, \mathrm dA_k \tau(S_{k-\gamma}) \frac {a \, \mathrm dV_\gamma}{S^2_{k-\gamma}} = \\
		= a^2 & \sum_{\gamma^* = 1}^{\Gamma} \sigma T^4_{\gamma^*} \int_{V_\gamma} \int_{V_{\gamma^*}} \frac{\tau(S_{\gamma^* - \gamma}) \, \mathrm dV_{\gamma^*} \, \mathrm dV_\gamma}{\pi S^2_{\gamma^* - \gamma}} + \\
		+ a & \sum_{k=1}^{N} q_{0, k} \int_{V_\gamma} \int_{A_k} \frac{\cos{\beta_k}}{\pi S^2_{k-\gamma}} \tau(S_{k-\gamma}) \, \mathrm dA_k \,\mathrm dV_\gamma.
	\end{aligned}
\end{equation}

Предполагается, что $a$ — постоянная величина по всему
замкнутому объему и что объемы $V_\gamma$ и все $V_{\gamma^*}$ являются изотермическими.
Величина $q_{0, k}$, как обычно, принимается постоянной на
поверхности $A_k$.

Определим взаимную поверхность обмена излучением между
поверхностью и газом в виде

\begin{equation}
	\label{eqn:ziegel-17-108}
	\overline{s_k g_\gamma} \equiv \frac a\pi \int_{V_\gamma} \int_{A_k} \frac{\cos{\beta_k}}{S^2_{k-\gamma}} \tau(S_{k-\gamma}) \,\mathrm dA_k \,\mathrm dV_\gamma.
\end{equation}

Сравнение \eqref{eqn:ziegel-17-108} и \eqref{eqn:ziegel-17-97} показывает, что существует
соотношение взаимности между взаимными поверхностями обмена
поверхность~— газ и газ~— поверхность

\begin{equation}
	\overline{s_k g_\gamma} = \overline{g_\gamma s_k}.
\end{equation}

Определим теперь взаимную поверхность обмена излучением
между двумя газами в виде

\begin{equation}
	\label{eqn:ziegel-17-110}
	\overline{g_{\gamma^*} g_\gamma} \equiv \frac{a^2}{\pi} \int_{V_\gamma} \int_{V_{\gamma^*}} \frac{\tau(S_{\gamma^*-\gamma}) \,\mathrm dV_{\gamma^*} \,\mathrm dV_\gamma}{S^2_{\gamma^*-\gamma}}.
\end{equation}

\noindent Подставляя \eqref{eqn:ziegel-17-108} — \eqref{eqn:ziegel-17-110} в \eqref{eqn:ziegel-17-107}, получим

\begin{equation}
	\label{eqn:ziegel-17-111}
	4a\sigma T^4_\gamma V_\gamma = \sum_{\gamma^* = 1}^{\Gamma} \sigma T^4_{\gamma^*} \overline{g_{\gamma^*} g_\gamma} + \sum_{k=1}^{N} q_{0, k} \overline{g_\gamma s_k}.
\end{equation}

\noindent Величины $\overline{g_{\gamma^*} g_\gamma}$ также табулированы [15], так что уравнение
\eqref{eqn:ziegel-17-111} для каждого объема $V_\gamma$ дает дополнительную систему
из $\Gamma$ уравнений, необходимую для вычисления распределения
температуры в газе.

% убрал пару абзацев

Описанный здесь метод расчета был развит далее Хоттелем
и др. [1, 2, 15]. При этом имеется возможность приближенным,
но доступным способом учесть спектральную зависимость свойств
газа. Изменения этих свойств в зависимости от положения в
замкнутом объеме учитываются путем определения соответствующего
среднего коэффициента поглощения между каждой серией зон.
Эйнштейн [18, 19] модифицировал коэффициенты $\overline{gs}$ и $\overline{gg}$ с целью
достижения большей точности расчетов при наличии больших
градиентов. Все эти приближенные методы становятся сложными
при наличии сильной зависимости коэффициента поглощения
от температуры.

\subsection{Метод обобщённых угловых коэффициентов}

\subsection{Метод дискретных ординат}

\subsection{Дискретно-лучевой метод}

\def\wA{5mm}
\def\wB{27mm}
\def\wC{23mm}
\def\wD{23mm}
\def\wE{23mm}
\def\wF{23mm}
\def\wG{23mm}
\def\wH{23mm}
\def\wI{22mm}
\def\wJ{22mm}
\begin{FixLineStretch}
\begin{sidewaystable}
	\small
	\caption{Сравнительный анализ существующих методов моделирования световых полей}
	\label{tbl:existing-modeling-methods}
	\begin{tabular}{|p{\wA}|p{\wB}|p{\wC}|p{\wD}|p{\wE}|p{\wF}|p{\wG}|p{\wH}|p{\wI}|p{\wJ}|}
		\hline
		\TableHeader{\wA}{№} & \TableHeader{\wB}{Метод}                                                                                & \TableHeader{\wC}{Учёт излучения плазмы источника из объёма} & \TableHeader{\wD}{Учёт неоднородного распределения параметров плазмы по объему} & \TableHeader{\wE}{Нахождение распределения поглощённой мощности по объёму плазмы} & \TableHeader{\wF}{Использование параллельных алгоритмов} & \TableHeader{\wG}{Возможность применения метода для построения замкнутых систем моделирования} & \TableHeader{\wH}{Универсальность подходов для расчёта систем с плазмой произвольной оптической плотности} & \TableHeader{\wI}{Отсутствие вероятностного розыгрыша лучей} & \TableHeader{\wJ}{Доступность ПМО и ЭВМ} \\ \hline
		1                    & \everypar{\hspace*{0pt}} Зональный \cite{radiation-heat-transfer}                                       & Нет                                                          & Нет                                                                             & Нет                                                                               & Да                                                       & Нет                                                                                            & Нет                                                                                                        & Да                                                           & Нет                                      \\ \hline
		2                    & \everypar{\hspace*{0pt}} Обобщённых угловых коэффициентов \cite{encyclopedia-of-low-temperature-plasma} & Нет                                                          & Нет                                                                             & Нет                                                                               & Да                                                       & Нет                                                                                            & Нет                                                                                                        & Да                                                           & Нет                                      \\ \hline
		3                    & \everypar{\hspace*{0pt}} Дискретных ординат \cite{surzhikov}                                            & Да                                                           & Нет                                                                             & Нет                                                                               & Да                                                       & Да                                                                                             & Нет                                                                                                        & Да                                                           & Нет                                      \\ \hline
		4                    & \everypar{\hspace*{0pt}} Дискретно-лучевой \cite{gradov-dissertation}                                   & Да                                                           & Да                                                                              & Нет                                                                               & Нет                                                      & Да                                                                                             & Да                                                                                                         & Нет                                                          & Да                                       \\ \hline
	\end{tabular}
\end{sidewaystable}
\end{FixLineStretch}
\let\wJ\relax
\let\wI\relax
\let\wH\relax
\let\wG\relax
\let\wF\relax
\let\wE\relax
\let\wD\relax
\let\wC\relax
\let\wB\relax
\let\wA\relax

\section{Формализация задачи}


\let\wF\relax