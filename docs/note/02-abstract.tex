%% ГОСТ 7.32-2017
%% 5.3 Реферат
%%
%% 5.3.1 Общие требования к реферату отчета о НИР — по ГОСТ 7.9.
%%
%% 5.3.2 Реферат должен содержать:
%% — сведения об общем объеме отчета, количестве книг отчета, иллюстраций, таблиц, использованных источников, приложений;
%% — перечень ключевых слов;
%% — текст реферата.
%%
%% 5.3.2.1 Перечень ключевых слов должен включать от 5 до 15 слов или словосочетаний из текста отчета, которые в наибольшей мере характеризуют его содержание и обеспечивают возможность информационного поиска.
%%
%% 5.3.2.2 Текст реферата должен отражать:
%% — объект исследования или разработки;
%% — цель работы;
%% — методы или методологию проведения работы;
%% — результаты работы и их новизну;
%% — область применения результатов;
%% — рекомендации по внедрению или итоги внедрения результатов НИР;
%% — экономическую эффективность или значимость работы;
%% — прогнозные предположения о развитии объекта исследования.
%%
%% Если отчет не содержит сведений по какой-либо из перечисленных структурных частей реферата, то в тексте реферата она опускается, при этом последовательность изложения сохраняется.
%%
%% Оптимальный объем текста реферата — 850 печатных знаков, но не более одной страницы машинописного текста.
%% Реферат следует оформлять в соответствии с 6.12.
%%
%% 5.3.3 Примеры составления рефератов к отчету о НИР приведены в приложении В.

\StructuralElement{Реферат}

% TODO(a.kerimov): Update.
Расчетно-пояснительная записка содержит 82~страницы, 4~части, 19~рисунков, 4~таблицы, 34~источника, 5~листингов и~1~приложение.

\Accent{Научная новизна}

\begin{enumerate}
	\item Разработан метод расчета распространения излучения в системе плазма-осветитель-приемник с учетом оптической неоднородности источника излучения и других поглощающе-излучающих элементов системы.

	\item Разработан метод нахождения распределения удельной мощности поглощения излучения в плазме и других элементах, в ходе многократных проходов света в системе сред и поверхностей.

	\item Разработан параллельный алгоритм реализации метода, свободный от вероятностного розыгрыша пути луча в точках рассеяния на диффузных поверхностях или преломления с учетом реальных индикатрис процессов.

	\item Получены данные о точности дифференциальных приближений, широко применяемых для расчета параметров источников излучения в задачах моделирования и проектирования изделий.
\end{enumerate}
