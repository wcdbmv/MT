%% ГОСТ 7.32-2017
%% 5.11 Приложения
%%
%% 5.11.1 В приложения рекомендуется включать материалы, дополняющие текст отчета, связанные с выполненной НИР, если они не могут быть включены в основную часть.
%%
%% В приложения могут быть включены:
%% - дополнительные материалы к отчету;
%% - промежуточные математические доказательства и расчеты;
%% - таблицы вспомогательных цифровых данных;
%% - протоколы испытаний;
%% - заключение метрологической экспертизы;
%% - инструкции, методики, описания алгоритмов и программ, разработанных в процессе выполнения НИР;
%% - иллюстрации вспомогательного характера;
%% - копии технического задания на НИР, программы работ или другие исходные документы для выполнения НИР;
%% - протокол рассмотрения результатов выполненной НИР на научно-техническом совете;
%% - акты внедрения результатов НИР или их копии;
%% - копии охранных документов.
%%
%% 5.11.2 Приложения к отчету о НИР, в составе которых предусмотрено проведение патентных исследований, могут быть включены в отчет о патентных исследованиях, оформленный по ГОСТ 15.011, библиографический список публикаций и патентных документов, полученных в результате выполнения НИР, который должен быть оформлен по ГОСТ 7.1, ГОСТ 7.80, ГОСТ 7.82.
%%
%% 5.11.3 Приложения оформляются в соответствии с 6.17.

%% Методические указания к выполнению, оформлению и защите выпускной квалификационной работы бакалавра
%% 2.12 Приложения
%%
%% Приложения состоят из вспомогательного материала, на который в основной части бакалаврской работы имеются ссылки.
%% Приложением оформляют различные схемы, листинг программ, наборы тестов и др.
%%
%% В тексте РПЗ на все приложения должны быть даны ссылки.

\chapter*{ПРИЛОЖЕНИЕ А}\label{toc:attachment-a}
\titleformat{\section}{\bigsize\centering\bfseries}{\thesection}{}{}{}{}
\section*{ЗАГОЛОВОК ПРИЛОЖЕНИЯ}
\addcontentsline{toc}{chapter}{ПРИЛОЖЕНИЕ А ЗАГОЛОВОК ПРИЛОЖЕНИЯ}
