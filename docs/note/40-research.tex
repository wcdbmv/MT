%% Методические указания к выполнению, оформлению и защите выпускной квалификационной работы бакалавра
%% 2.7 Экспериментальный раздел
%%
%% Данный раздел содержит описание проведенных экспериментов и их результаты.
%% Должно быть обязательно указано, какую цель ставил перед собой автор работы при планировании экспериментов, какие предположения/гипотезы он надеялся подтвердить и/или опровергнуть с их помощью.
%% Результаты оформляются в виде графиков, диаграмм и/или таблиц.
%%
%% Здесь же может быть проведено качественное и количественное сравнение с аналогами.
%%
%% Рекомендуемый объем экспериментального раздела 10—15 страниц.

\chapter{Исследовательский раздел}

\section{Спектральная оптическая плотность плазмы}

\begin{figure}[ht]
	\noindent\begin{tikzpicture}
		\begin{axis}[
			xlabel = {$\nu$},
			ylabel = {$\tau$},
			xmin = 0, xmax = 2.8e+15,
			grid = both,
			width = \linewidth,
			height = 0.5\linewidth,
		]
			\addplot table [x=nu, y=tau]{inc/data/optical_density.dat};
		\end{axis}
	\end{tikzpicture}

	\noindent\begin{tikzpicture}
		\begin{axis}[
			xlabel = {$\nu$},
			ylabel = {$\tau$},
			domain=1e+14:4e+14,
			xmin = 1e+14, xmax = 7e+14,
			grid = both,
			width = 0.485\linewidth,
			height = 0.5\linewidth,
		]
			\addplot table [x=nu, y=tau]{inc/data/optical_density.dat};
	\end{axis}
	\end{tikzpicture}\begin{tikzpicture}
		\begin{semilogyaxis}[
			xlabel = {$\nu$},
			ylabel = {$\tau$},
			xmin = 19e+14, xmax = 28e+14,
			grid = both,
			width = 0.485\linewidth,
			height = 0.5\linewidth,
		]
			\addplot table [x=nu, y=tau]{inc/data/optical_density.dat};
		\end{semilogyaxis}
	\end{tikzpicture}
	\captionsetup{justification=centering}
	\caption{Спектральная оптическая плотность разрядной плазмы}
	\label{plt:optical-density}
\end{figure}

На рисунке \ref{plt:optical-density} представлена спектральная оптическая плотность разрядной плазмы радиусом $R = 0,35$ см, осевой и краевой температурами $T_0 = 10000$ К, $T_w = 2000$ К, показателем степени $m = 4$ (см. \eqref{eqn:xenon-temperature}) и зеркальным отражателем на поверхности плазмы с коэффициентом отражения $\rho = 0,95$, рассчитываемая по формуле:
\begin{equation}
	\tau = \int\displaylimits_0^R k_{\text{погл}} \,\mathrm dr.
\end{equation}

\begin{figure}[ht]
	\noindent\begin{tikzpicture}
		\begin{axis}[
			xlabel = {$z$},
			ylabel = {$I \;\; [\text{Вт}/\text{см}^2]$},
			xmin = 0, xmax = 1.025,
			yticklabel style={/pgf/number format/.cd,fixed,precision=2},
			grid = both,
			width = 0.965\linewidth,
			height = 0.5\linewidth,
		]
			%\addplot table [x=zmax, y=I2tau57]{inc/data/Xe_R=0.35_d_nu.dat};
			%\addlegendentry{$\tau_{57} = 0,389$}
			%\addplot table [x=zmax, y=I2tau58]{inc/data/Xe_R=0.35_d_nu.dat};
			%\addlegendentry{$\tau_{58} = 0,393$}
			%\addplot table [x=zmax, y=I2tau59]{inc/data/Xe_R=0.35_d_nu.dat};
			%\addlegendentry{$\tau_{59} = 0,395$}
			%\addplot table [x=zmax, y=I2tau60]{inc/data/Xe_R=0.35_d_nu.dat};
			%\addlegendentry{$\tau_{60} = 0,356$}
			%\addplot table [x=zmax, y=I2tau61]{inc/data/Xe_R=0.35_d_nu.dat};
			%\addlegendentry{$\tau_{61} = 0,379$}
			%\addplot table [x=zmax, y=I2tau79]{inc/data/Xe_R=0.35_d_nu.dat};
			%\addlegendentry{$\tau_{79} = 0,66$}
			\addplot table [x=zmax, y=I2tau80]{inc/data/Xe_R=0.35_d_nu.dat};
			\addlegendentry{$\tau_{80} = 0,68$}
			\addplot table [x=zmax, y=I2tau81]{inc/data/Xe_R=0.35_d_nu.dat};
			\addlegendentry{$\tau_{81} = 0,69$}
			\addplot table [x=zmax, y=I2tau82]{inc/data/Xe_R=0.35_d_nu.dat};
			\addlegendentry{$\tau_{82} = 0,75$}
		\end{axis}
	\end{tikzpicture}

	\noindent\begin{tikzpicture}
		\begin{axis}[
			xlabel = {$z$},
			ylabel = {$I^* \;\; [\text{Вт}/\text{см}^3]$},
			xmin = 0, xmax = 1.025,
			grid = both,
			width = 0.98\linewidth,
			height = 0.5\linewidth,
		]
			%\addplot table [x=zmax, y=I3tau57]{inc/data/Xe_R=0.35_d_nu.dat};
			%\addlegendentry{$\tau_{57} = 0,389$}
			%\addplot table [x=zmax, y=I3tau58]{inc/data/Xe_R=0.35_d_nu.dat};
			%\addlegendentry{$\tau_{58} = 0,393$}
			%\addplot table [x=zmax, y=I3tau59]{inc/data/Xe_R=0.35_d_nu.dat};
			%\addlegendentry{$\tau_{59} = 0,395$}
			%\addplot table [x=zmax, y=I3tau60]{inc/data/Xe_R=0.35_d_nu.dat};
			%\addlegendentry{$\tau_{60} = 0,356$}
			%\addplot table [x=zmax, y=I3tau61]{inc/data/Xe_R=0.35_d_nu.dat};
			%\addlegendentry{$\tau_{61} = 0,379$}
			%\addplot table [x=zmax, y=I3tau79]{inc/data/Xe_R=0.35_d_nu.dat};
			%\addlegendentry{$\tau_{79} = 0,66$}
			\addplot table [x=zmax, y=I3tau80]{inc/data/Xe_R=0.35_d_nu.dat};
			\addlegendentry{$\tau_{80} = 0,68$}
			\addplot table [x=zmax, y=I3tau81]{inc/data/Xe_R=0.35_d_nu.dat};
			\addlegendentry{$\tau_{81} = 0,69$}
			\addplot table [x=zmax, y=I3tau82]{inc/data/Xe_R=0.35_d_nu.dat};
			\addlegendentry{$\tau_{82} = 0,75$}
		\end{axis}
	\end{tikzpicture}
	\captionsetup{justification=centering}
	\caption{Поглощённая мощность плазмы, $R = 0,35$ см, $T_0 = 10000$ К, $T_w = 2000$ К, $m = 4$, $\rho = 0,95$}
	\label{plt:d_nu}
\end{figure}

На рисунке \ref{plt:d_nu} представлена зависимость распределения поглощённой мощности по объёму разрядной плазмы при различных диапазонах частот.

Диапазоны частот, соответствующие критически малым значениям оптической плотности, являются по сути прозрачными для светового излучения, а в диапазонах частот, соответствующих большим значениям оптической плотности, плазма выступает в качестве чёрного газа.
% В дальнейшем расчёты поглощённой мощности по объёму будут осуществляться в диапазонах $[0,36232\cdot10^{15}, 0,36240\cdot10^{15}]$ c $\tau_{81} = 0,691569$ и $[2,38460\cdot10^{15}, 2,38960\cdot10^{15}]$ с $\tau_{170} = 2,507276$.

% TODO(a.kerimov): Приложение

\section{Влияние коэффициента отражения}

\begin{figure}[ht]
	\noindent\begin{tikzpicture}
		\begin{axis}[
			xlabel = {$z$},
			ylabel = {$I \;\; [\text{Вт}/\text{см}^2]$},
			xmin = 0, xmax = 1.025,
			grid = both,
			width = \linewidth,
			height = 0.5\linewidth,
		]
			\addplot table [x=zmax, y=I2T10m4r55]{inc/data/Xe_R=0.35_rho.dat};
			\addlegendentry{$\rho = 0,55$}
			\addplot table [x=zmax, y=I2T10m4r70]{inc/data/Xe_R=0.35_rho.dat};
			\addlegendentry{$\rho = 0,70$}
			\addplot table [x=zmax, y=I2T10m4r80]{inc/data/Xe_R=0.35_rho.dat};
			\addlegendentry{$\rho = 0,80$}
			\addplot table [x=zmax, y=I2T10m4r90]{inc/data/Xe_R=0.35_rho.dat};
			\addlegendentry{$\rho = 0,90$}
			\addplot table [x=zmax, y=I2T10m4r95]{inc/data/Xe_R=0.35_rho.dat};
			\addlegendentry{$\rho = 0,95$}
			\addplot table [x=zmax, y=I2T10m4r98]{inc/data/Xe_R=0.35_rho.dat};
			\addlegendentry{$\rho = 0,98$}
		\end{axis}
	\end{tikzpicture}

	\noindent\begin{tikzpicture}
		\begin{axis}[
			xlabel = {$z$},
			ylabel = {$I^* \;\; [\text{Вт}/\text{см}^3]$},
			xmin = 0, xmax = 1.025,
			grid = both,
			width = 0.98\linewidth,
			height = 0.5\linewidth,
		]
			\addplot table [x=zmax, y=I3T10m4r70]{inc/data/Xe_R=0.35_rho.dat};
			\addlegendentry{$\rho = 0,70$}
			\addplot table [x=zmax, y=I3T10m4r80]{inc/data/Xe_R=0.35_rho.dat};
			\addlegendentry{$\rho = 0,80$}
			\addplot table [x=zmax, y=I3T10m4r90]{inc/data/Xe_R=0.35_rho.dat};
			\addlegendentry{$\rho = 0,90$}
			\addplot table [x=zmax, y=I3T10m4r95]{inc/data/Xe_R=0.35_rho.dat};
			\addlegendentry{$\rho = 0,95$}
			\addplot table [x=zmax, y=I3T10m4r98]{inc/data/Xe_R=0.35_rho.dat};
			\addlegendentry{$\rho = 0,98$}
		\end{axis}
	\end{tikzpicture}
	\captionsetup{justification=centering}
	\caption{Поглощённая мощность плазмы, $\Delta\nu_{81} = [0,36232\cdot10^{15}, 0,36240\cdot10^{15}]$ $R = 0,35$ см, $T_0 = 10000$ К, $T_w = 2000$ К, $m = 4$}
	\label{plt:rho}
\end{figure}

На рисунке \ref{plt:rho} представлена зависимость распределения поглощённой мощности по объёму разрядной плазмы при переменном коэффициенте $\rho$ зеркального отражения на поверхности.

При $\rho = 0,55$ доля поглощённой мощности в системе при заданных параметрах упала более, чем в два раза по сравнению со случаем при $\rho = 0,98$.
Это связано с тем, что плазма не является абсолютно поглощающей средой, и в разряде реализуется режим многократного прохождения излучения через плазму до его полного выхода из системы в процессе отражения от границы.

С увеличением коэффициента отражения количество таких проходов растет, увеличивается доля поглощённой мощности.
При этом для ксеноновой плазмы с её высокой излучательной способностью вид распределения практически не меняется.

\section{Влияние распределения температуры}

\begin{figure}[H]
	\noindent\begin{tikzpicture}
		\begin{axis}[
			xlabel = {$z$},
			ylabel = {$T \;\; [\text{К}]$},
			xmin = 0, xmax = 1.0,
			domain=0:1,
			grid = both,
			width = 0.965\linewidth,
			height = 0.5\linewidth,
			legend style={at={(0.02,0.02)},anchor=south west},
		]
			\addplot {10000 + (2000 - 10000)*x*x};
			\addlegendentry{$T_0 = 10000, T_w = 2000, m = 2$}
			\addplot {10000 + (2000 - 10000)*x*x*x*x};
			\addlegendentry{$T_0 = 10000, T_w = 2000, m = 4$}
			\addplot {10000 + (2000 - 10000)*x*x*x*x*x*x};
			\addlegendentry{$T_0 = 10000, T_w = 2000, m = 6$}
			\addplot {10000 + (2000 - 10000)*x*x*x*x*x*x*x*x};
			\addlegendentry{$T_0 = 10000, T_w = 2000, m = 8$}
			\addplot {12000 + (2000 - 12000)*x*x*x*x};
			\addlegendentry{$T_0 = 12000, T_w = 2000, m = 4$}
			\addplot {10000 + (4000 - 10000)*x*x*x*x};
			\addlegendentry{$T_0 = 10000, T_w = 4000, m = 4$}
		\end{axis}
	\end{tikzpicture}
	\captionsetup{justification=centering}
	\caption{Температурное распределение по модели $T(z) = T_0 + (T_w - T_0)z^m$}
	\label{plt:temperature}
\end{figure}

На рисунке \ref{plt:temperature} представлена модель температурного профиля разрядного ксенона. В этой модели распределения коэффициент $m$ отвечает за продолжительность плоского фронта осевой температуры в основной части разряда, отдаляя крутое падение у стенки.

На рисунке \ref{plt:m} представлена зависимость распределения поглощённой мощности по объёму разрядной плазмы при переменном показателе степени $m$ убывания абсолютной температуры.
Видно, что с ростом коэффициента $m$ максимум распределений $I$, $I^*$ смещается ближе к поверхности ксенона, а суммарная мощность поглощённого распределения возрастает.
Это связано с тем, что взлёт общей температуры разряда увеличивает излучающую способность материала (см. \eqref{eqn:intensity-plank}) намного сильнее, чем соответствующее уменьшение коэффициента оптического поглощения в сильно нагретой среде.

\begin{figure}[ht]
	\noindent\begin{tikzpicture}
		\begin{axis}[
			xlabel = {$z$},
			ylabel = {$I \;\; [\text{Вт}/\text{см}^2]$},
			xmin = 0, xmax = 1.025,
			yticklabel style={/pgf/number format/.cd,fixed,precision=2},
			grid = both,
			width = 0.965\linewidth,
			height = 0.5\linewidth,
		]
			\addplot table [x=zmax, y=I2T10m2r95]{inc/data/Xe_R=0.35_m.dat};
			\addlegendentry{$m = 2$}
			\addplot table [x=zmax, y=I2T10m3r95]{inc/data/Xe_R=0.35_m.dat};
			\addlegendentry{$m = 3$}
			\addplot table [x=zmax, y=I2T10m4r95]{inc/data/Xe_R=0.35_m.dat};
			\addlegendentry{$m = 4$}
			\addplot table [x=zmax, y=I2T10m5r95]{inc/data/Xe_R=0.35_m.dat};
			\addlegendentry{$m = 5$}
			\addplot table [x=zmax, y=I2T10m6r95]{inc/data/Xe_R=0.35_m.dat};
			\addlegendentry{$m = 6$}
			\addplot table [x=zmax, y=I2T10m7r95]{inc/data/Xe_R=0.35_m.dat};
			\addlegendentry{$m = 7$}
			\addplot table [x=zmax, y=I2T10m8r95]{inc/data/Xe_R=0.35_m.dat};
			\addlegendentry{$m = 8$}
		\end{axis}
	\end{tikzpicture}

	\noindent\begin{tikzpicture}
		\begin{axis}[
			xlabel = {$z$},
			ylabel = {$I^* \;\; [\text{Вт}/\text{см}^3]$},
			xmin = 0, xmax = 1.025,
			grid = both,
			width = 0.98\linewidth,
			height = 0.5\linewidth,
		]
			\addplot table [x=zmax, y=I3T10m2r95]{inc/data/Xe_R=0.35_m.dat};
			\addlegendentry{$m = 2$}
			\addplot table [x=zmax, y=I3T10m3r95]{inc/data/Xe_R=0.35_m.dat};
			\addlegendentry{$m = 3$}
			\addplot table [x=zmax, y=I3T10m4r95]{inc/data/Xe_R=0.35_m.dat};
			\addlegendentry{$m = 4$}
			\addplot table [x=zmax, y=I3T10m5r95]{inc/data/Xe_R=0.35_m.dat};
			\addlegendentry{$m = 5$}
			\addplot table [x=zmax, y=I3T10m6r95]{inc/data/Xe_R=0.35_m.dat};
			\addlegendentry{$m = 6$}
			\addplot table [x=zmax, y=I3T10m7r95]{inc/data/Xe_R=0.35_m.dat};
			\addlegendentry{$m = 7$}
			\addplot table [x=zmax, y=I3T10m8r95]{inc/data/Xe_R=0.35_m.dat};
			\addlegendentry{$m = 8$}
		\end{axis}
	\end{tikzpicture}
	\captionsetup{justification=centering}
	\caption{Поглощённая мощность плазмы, $\Delta\nu_{81} = [0,36232\cdot10^{15}, 0,36240\cdot10^{15}]$ $R = 0,35$ см, $T_0 = 10000$ К, $T_w = 2000$ К, $\rho = 0,95$}
	\label{plt:m}
\end{figure}

\begin{figure}[ht]
	\noindent\begin{tikzpicture}
		\begin{axis}[
			xlabel = {$z$},
			ylabel = {$I \;\; [\text{Вт}/\text{см}^2]$},
			xmin = 0, xmax = 1.025,
			yticklabel style={/pgf/number format/.cd,fixed,precision=2},
			grid = both,
			width = 0.965\linewidth,
			height = 0.5\linewidth,
		]
			\addplot table [x=zmax, y=I2T8m4r95]{inc/data/Xe_R=0.35_T0.dat};
			\addlegendentry{$T_0 = 8000 \; \text{К}$}
			\addplot table [x=zmax, y=I2T9m4r95]{inc/data/Xe_R=0.35_T0.dat};
			\addlegendentry{$T_0 = 9000 \; \text{К}$}
			\addplot table [x=zmax, y=I2T10m4r95]{inc/data/Xe_R=0.35_T0.dat};
			\addlegendentry{$T_0 = 10000 \; \text{К}$}
			\addplot table [x=zmax, y=I2T11m4r95]{inc/data/Xe_R=0.35_T0.dat};
			\addlegendentry{$T_0 = 11000 \; \text{К}$}
			\addplot table [x=zmax, y=I2T12m4r95]{inc/data/Xe_R=0.35_T0.dat};
			\addlegendentry{$T_0 = 12000 \; \text{К}$}
		\end{axis}
	\end{tikzpicture}

	\noindent\begin{tikzpicture}
		\begin{axis}[
			xlabel = {$z$},
			ylabel = {$I^* \;\; [\text{Вт}/\text{см}^3]$},
			xmin = 0, xmax = 1.025,
			grid = both,
			width = 0.98\linewidth,
			height = 0.5\linewidth,
		]
			\addplot table [x=zmax, y=I3T8m4r95]{inc/data/Xe_R=0.35_T0.dat};
			\addlegendentry{$T_0 = 8000 \; \text{К}$}
			\addplot table [x=zmax, y=I3T9m4r95]{inc/data/Xe_R=0.35_T0.dat};
			\addlegendentry{$T_0 = 9000 \; \text{К}$}
			\addplot table [x=zmax, y=I3T10m4r95]{inc/data/Xe_R=0.35_T0.dat};
			\addlegendentry{$T_0 = 10000 \; \text{К}$}
			\addplot table [x=zmax, y=I3T11m4r95]{inc/data/Xe_R=0.35_T0.dat};
			\addlegendentry{$T_0 = 11000 \; \text{К}$}
			\addplot table [x=zmax, y=I3T12m4r95]{inc/data/Xe_R=0.35_T0.dat};
			\addlegendentry{$T_0 = 12000 \; \text{К}$}
		\end{axis}
	\end{tikzpicture}
	\captionsetup{justification=centering}
	\caption{Поглощённая мощность плазмы, $\Delta\nu_{81} = [0,36232\cdot10^{15}, 0,36240\cdot10^{15}]$ $R = 0,35$ см, $T_w = 2000$ К, $m = 4$, $\rho = 0,95$}
	\label{plt:t0}
\end{figure}

На рисунках \ref{plt:t0} и \ref{plt:tw} представлена зависимость распределения поглощённой мощности по объёму разрядной плазмы при переменном значении осевой и радиальной абсолютных температур $T_0$ и $T_w$ соответственно.

Видно, что тенденция изменения $I$ и $I^*$ сохраняется: рост общей температуры разряда влияет на увеличение пика распределения поглощённой мощности и смещает его ближе к поверхности ксенонового излучателя.

\begin{figure}[H]
	\noindent\begin{tikzpicture}
		\begin{axis}[
			xlabel = {$z$},
			ylabel = {$I \;\; [\text{Вт}/\text{см}^2]$},
			xmin = 0, xmax = 1.025,
			yticklabel style={/pgf/number format/.cd,fixed,precision=2},
			grid = both,
			width = 0.965\linewidth,
			height = 0.5\linewidth,
			legend style={at={(0.02,0.98)},anchor=north west},
		]
			\addplot table [x=zmax, y=I2T12w2m4r95]{inc/data/Xe_R=0.35_Tw.dat};
			\addlegendentry{$T_w = 2000 \; \text{К}$}
			\addplot table [x=zmax, y=I2T12w3m4r95]{inc/data/Xe_R=0.35_Tw.dat};
			\addlegendentry{$T_w = 3000 \; \text{К}$}
			\addplot table [x=zmax, y=I2T12w4m4r95]{inc/data/Xe_R=0.35_Tw.dat};
			\addlegendentry{$T_w = 4000 \; \text{К}$}
			\addplot table [x=zmax, y=I2T12w5m4r95]{inc/data/Xe_R=0.35_Tw.dat};
			\addlegendentry{$T_w = 5000 \; \text{К}$}
			\addplot table [x=zmax, y=I2T12w6m4r95]{inc/data/Xe_R=0.35_Tw.dat};
			\addlegendentry{$T_w = 6000 \; \text{К}$}
			\addplot table [x=zmax, y=I2T12w7m4r95]{inc/data/Xe_R=0.35_Tw.dat};
			\addlegendentry{$T_w = 7000 \; \text{К}$}
			\addplot table [x=zmax, y=I2T12w8m4r95]{inc/data/Xe_R=0.35_Tw.dat};
			\addlegendentry{$T_w = 8000 \; \text{К}$}
		\end{axis}
	\end{tikzpicture}

	\noindent\begin{tikzpicture}
		\begin{axis}[
			xlabel = {$z$},
			ylabel = {$I^* \;\; [\text{Вт}/\text{см}^3]$},
			xmin = 0, xmax = 1.025,
			grid = both,
			width = 0.98\linewidth,
			height = 0.5\linewidth,
			legend style={at={(0,1)},anchor=north west},
		]
			\addplot table [x=zmax, y=I3T12w2m4r95]{inc/data/Xe_R=0.35_Tw.dat};
			\addlegendentry{$T_w = 2000 \; \text{К}$}
			\addplot table [x=zmax, y=I3T12w3m4r95]{inc/data/Xe_R=0.35_Tw.dat};
			\addlegendentry{$T_w = 3000 \; \text{К}$}
			\addplot table [x=zmax, y=I3T12w4m4r95]{inc/data/Xe_R=0.35_Tw.dat};
			\addlegendentry{$T_w = 4000 \; \text{К}$}
			\addplot table [x=zmax, y=I3T12w5m4r95]{inc/data/Xe_R=0.35_Tw.dat};
			\addlegendentry{$T_w = 5000 \; \text{К}$}
			\addplot table [x=zmax, y=I3T12w6m4r95]{inc/data/Xe_R=0.35_Tw.dat};
			\addlegendentry{$T_w = 6000 \; \text{К}$}
			\addplot table [x=zmax, y=I3T12w7m4r95]{inc/data/Xe_R=0.35_Tw.dat};
			\addlegendentry{$T_w = 7000 \; \text{К}$}
			\addplot table [x=zmax, y=I3T12w8m4r95]{inc/data/Xe_R=0.35_Tw.dat};
			\addlegendentry{$T_w = 8000 \; \text{К}$}
		\end{axis}
	\end{tikzpicture}
	\captionsetup{justification=centering}
	\caption{Поглощённая мощность плазмы, $\Delta\nu_{81} = [0,36232\cdot10^{15}, 0,36240\cdot10^{15}]$ $R = 0,35$ см, $T_0 = 12000$ К, $m = 4$, $\rho = 0,95$}
	\label{plt:tw}
\end{figure}

\section{Влияние количества лучей}

На рисунке \ref{plt:n-points} представлена зависимость распределения поглощённой мощности по объёму разрядной плазмы при переменном количестве лучей $N = N_{\text{меридианов}} \cdot N_{\text{широт}}$.

Даже при $N = 500$ в графике функции $I^*$ наблюдается небольшое, но бросающееся в глаз отклонение во втором квазистационарном слое ($\Delta z = [0,025; 0,05]$) разряда ксенона.

\begin{figure}[H]
	\noindent\begin{tikzpicture}
		\begin{axis}[
			xlabel = {$z$},
			ylabel = {$I \;\; [\text{Вт}/\text{см}^2]$},
			xmin = 0, xmax = 1.025,
			grid = both,
			width = \linewidth,
			height = 0.5\linewidth,
		]
			\addplot table [x=zmax, y=I2n50]{inc/data/Xe_R=0.35_Npoints.dat};
			\addlegendentry{$N = 50$}
			\addplot table [x=zmax, y=I2n100]{inc/data/Xe_R=0.35_Npoints.dat};
			\addlegendentry{$N = 100$}
			\addplot table [x=zmax, y=I2n200]{inc/data/Xe_R=0.35_Npoints.dat};
			\addlegendentry{$N = 200$}
			\addplot table [x=zmax, y=I2n500]{inc/data/Xe_R=0.35_Npoints.dat};
			\addlegendentry{$N = 500$}
			\addplot table [x=zmax, y=I2n4500]{inc/data/Xe_R=0.35_Npoints.dat};
			\addlegendentry{$N = 4500$}
		\end{axis}
	\end{tikzpicture}

	\noindent\begin{tikzpicture}
		\begin{axis}[
			xlabel = {$z$},
			ylabel = {$I^* \;\; [\text{Вт}/\text{см}^3]$},
			xmin = 0, xmax = 1.025,
			grid = both,
			width = 0.98\linewidth,
			height = 0.5\linewidth,
		]
			\addplot table [x=zmax, y=I3n50]{inc/data/Xe_R=0.35_Npoints.dat};
			\addlegendentry{$N = 50$}
			\addplot table [x=zmax, y=I3n100]{inc/data/Xe_R=0.35_Npoints.dat};
			\addlegendentry{$N = 100$}
			\addplot table [x=zmax, y=I3n200]{inc/data/Xe_R=0.35_Npoints.dat};
			\addlegendentry{$N = 200$}
			\addplot table [x=zmax, y=I3n500]{inc/data/Xe_R=0.35_Npoints.dat};
			\addlegendentry{$N = 500$}
			\addplot table [x=zmax, y=I3n4500]{inc/data/Xe_R=0.35_Npoints.dat};
			\addlegendentry{$N = 4500$}
		\end{axis}
	\end{tikzpicture}
	\captionsetup{justification=centering}
	\caption{Поглощённая мощность плазмы, $\Delta\nu_{81} = [0,36232\cdot10^{15}, 0,36240\cdot10^{15}]$ $R = 0,35$ см, $T_0 = 10000$ К, $T_w = 2000$ К, $m = 4$, переменное количество лучей $N$}
	\label{plt:n-points}
\end{figure}

Помимо количества лучей $N$, на гладкость функции распределения поглощённой мощности влияют так же количество разбиений $N_p$ (см. листинг \ref{alg:initialization}) и значение критической интенсивности $I_{\text{крит}}$ (см. рисунок \ref{img:ray-tracing}).
При заданной конфигурации системы приемлемая гладкость распределения $I^*$ достигается при $N = 4500$.

\pagebreak

\section{Влияние возврата излучения}

Эффект дополнительного подогрева плазмы и, как следствие, изменения её теплофизических и излучательных свойств происходит в результате возврата части излученной мощности обратно в плазменный столб.

Математическая модель строится для азимутально симметричного цилиндрического столба разряда, находящегося в условиях локального термодинамического равновесия, с учётом переноса излучения.
Соответствующая система уравнений включает уравнение энергии, закон Ома и математическую модель переноса излучения:

\begin{equation}
	\label{eqn:article-1}
	\frac1r\frac{\mathrm d}{\mathrm dr} \left( r \lambda(T) \frac{\mathrm dT}{\mathrm dr} \right) + \sigma(T)E^2 - \Div{\vec F = 0},
\end{equation}
\begin{equation}
	\label{eqn:article-2}
	\begin{gathered}
		\Div{\vec F} = \int\displaylimits_\nu \Div{\vec F_\nu} \,\mathrm d\nu, \\
		\Div{\vec F_\nu} = ck_{\nu}(u_{p\nu} - u_{\nu}) = ck_{\nu}u_{p\nu} - q_{s\nu}, \\
	\end{gathered}
\end{equation}
\begin{equation}
	\label{eqn:article-3}
	E = \frac{I}{\displaystyle 2\pi\int\displaylimits_0^R \sigma(T) r\,\mathrm dr}.
\end{equation}

В этих формулах: $T$~— температурное поле в разряде, $\lambda, \sigma, k_\nu$~— коэффициенты электропроводности, теплопроводности и спектральный коэффициент поглощения плазмы, $r, R$~— текущий радиус и внутренний радиус разрядной трубки, $\nu$~— частота излучения, $\vec F, \vec F_\nu, u_{p\nu}, u_{\nu}, q_{s\nu}$~— спектральный и интегральный потоки излучения, функция Планка, объёмная плотность излучения в плазме, объёмная мощность поглощенного излучения, $I, E$~— электрический ток и напряженность электрического поля.
Система дополняется соответствующими краевыми условиями \cite{gradov-dissertation}.

После расчёта радиального распределения поглощённой в плазме мощности $\displaystyle q_{s\nu} = \int\displaylimits_0^R I^* \,\mathrm dr$, фигурирующей в \eqref{eqn:article-2}, идёт определения интеграла по частоте от $q_{s\nu}$, затем выполняется итерация по расчёту температурного поля в разряде из уравнения энергии \eqref{eqn:article-1}.
Процедура повторяется до выполнения условия сходимости по температуре плазмы \cite{article-5-kalitkin}.

Обработка многочисленных вычислительных экспериментов \cite{gradov-dissertation, article-6-gradov} показала, что при решении частной задачи о влиянии возврата излучения на его теплофизические характеристики сложная процедура расчёта прохождения излучения в системе может быть заменена более простой моделью отражении излучения на границе плазмы с некоторым эффективным коэффициентом отражения $\rho$.
Данный коэффициент рассчитывается на основе сравнения мощности, поглощенной в плазме, вычисленной при этих двух подходах.
При этом расчёт переноса излучения в плазме может быть рассмотрен в рамках более простого диффузионного приближения \cite{gradov-dissertation, article-6-gradov, article-7-zeldovich} с краевым условием при $r = R$

\begin{equation}
	u_\nu = \frac{A}{k_\nu} \frac{1 + \rho}{1 - \rho} \frac{\mathrm du_\nu}{\mathrm dr},
\end{equation}

\noindent где $A$ — некоторая константа \cite{gradov-dissertation, article-6-gradov}.

В качестве исходных данных в рассматриваемой модели разряда задаются ток, диаметр разрядной трубки, давление наполнения, эффективный коэффициент отражения на границе, одинаковый для всех частот в интервале 100~нм~— 3~мкм, и все материальные функции плазмы \cite{article-8-gradov, article-9-gradov}.

На рисунке \ref{plt:article-1} продемонстрировано, что наличие отражения на границе плазмы сопровождается ростом общего уровня температур.
При этом для ксеноновой плазмы с её высокой излучательной способностью вид распределения практически не меняется, оставаясь плоским в основной области разряда с крутым фронтом падения у стенки.

При малых коэффициентах отражения изменения температуры практически не происходит.
При $\rho = 0,55$ осевая температура $T_0$ меняется всего на 3 \% по сравнению со случаем отсутствия отражения.
Это связано с тем, что плазма не является абсолютно поглощающей средой, и в разряде реализуется режим многократного прохождения излучения через плазму до его полного выхода из системы в процессе отражения от границы.

\begin{figure}[H]
	\noindent\begin{tikzpicture}
		\begin{axis}[
			xlabel = {$z$},
			ylabel = {$T \;\; [\text{К}]$},
			xmin = 0, xmax = 1.0,
			domain=0:1,
			samples=60,
			%samples at={0,0.05,...,0.9,0.91,...,1.05},
			grid = both,
			width = 0.965\linewidth,
			height = 0.5\linewidth,
			legend style={at={(0.02,0.02)},anchor=south west},
		]
			\addplot[thick, mark=none] {10400 + (2000 - 10400)*x^60 + 30*rand};
			\addlegendentry{$\rho = 0$}
			\addplot[densely dashed, thick, mark=none] {10700 + (2000 - 10700)*x^65 + 20*rand};
			\addlegendentry{$\rho = 0,55$}
			\addplot[loosely dashed, thick, mark=none] {10850 + (2000 - 10850)*x^70 + 20*rand};
			\addlegendentry{$\rho = 0,7$}
			\addplot[densely dashdotted, thick, mark=none] {11000 + (2000 - 11000)*x^75 + 20*rand};
			\addlegendentry{$\rho = 0,8$}
			\addplot[loosely dashdotted, thick, mark=none] {11500 + (2000 - 11500)*x^80 + 20*rand};
			\addlegendentry{$\rho = 0,9$}
			\addplot[dashdotdotted, thick, mark=none] {12500 + (2000 - 12500)*x^90 + 20*rand};
			\addlegendentry{$\rho = 0,97$}
		\end{axis}
	\end{tikzpicture}
	\captionsetup{justification=centering}
	\caption{Влияние коэффициента отражения $\rho$ на температурные распределения в разряде. $R=0,35$ см, $p=1,5$ МПа, $I=500$ A}
	\label{plt:article-1}
\end{figure}

\begin{figure}[H]
	\begin{center}
	\noindent\begin{tikzpicture}
		\begin{axis}[
			xlabel = {$\rho$},
			ylabel = {$T \;\; [\text{К}]$},
			xmin = 0, xmax = 1.0,
			domain=0:1,
			grid = both,
			width = 0.5\linewidth,
			height = 0.5\linewidth,
			legend style={at={(0.02,0.98)},anchor=north west},
		]
			\addplot[very thick, blue, mark=none] table [x=z, y=T0]{inc/data/article_2.dat};
			\addlegendentry{$T_0$}
			\addplot[very thick, loosely dashed, mark=none] table [x=z, y=T_avg]{inc/data/article_2.dat};
			\addlegendentry{$\langle T \rangle$}
		\end{axis}
	\end{tikzpicture}
	\end{center}
	\captionsetup{justification=centering}
	\caption{Зависимость осевой и средней температур в разряде от
		коэффициента отражения. Параметры разряда те же, что и на рис. \ref{plt:article-1}}
	\label{plt:article-2}
\end{figure}

С увеличением коэффициента отражения количество таких проходов растет, увеличивается доля поглощённой мощности и повышается темп роста температуры по мере возрастания $\rho$.
Последнее наглядно показывает рисунок \ref{plt:article-2}.

Отметим также, что при повышении температуры растет коэффициент поглощения плазмы \cite{article-8-gradov, article-9-gradov}, что является дополнительным фактором более высокого темпа нарастания температуры при приближении $\rho$ к единице.
При изменении $\rho$ от 0,55 до 0,80 осевая температура увеличивается на 4,5 \%, а при увеличении $\rho$ от 0,80 до 0,97 рост $T_0$ составляет уже 14 \%.

Отметим, что спектральное распределение лучистых потоков в открытом разряде и в составе осветительной системы отличаются и тем сильнее, чем выше температура, давление плазмы и, соответственно, коэффициент отражения.

\section*{Выводы}

Таким образом, в рамках проведенного исследования на основе математического моделирования получены количественные данные по влиянию возврата излучения на характеристики разряда.
Данные такого рода позволяют проводить оптимизацию электрических цепей и оценивать изменения, происходящие в спектрах излучения.
