%% Методические указания к выполнению, оформлению и защите выпускной квалификационной работы бакалавра
%% 2.7 Экспериментальный раздел
%%
%% Данный раздел содержит описание проведенных экспериментов и их результаты.
%% Должно быть обязательно указано, какую цель ставил перед собой автор работы при планировании экспериментов, какие предположения/гипотезы он надеялся подтвердить и/или опровергнуть с их помощью.
%% Результаты оформляются в виде графиков, диаграмм и/или таблиц.
%%
%% Здесь же может быть проведено качественное и количественное сравнение с аналогами.
%%
%% Рекомендуемый объем экспериментального раздела 10—15 страниц.

\chapter{Исследовательский раздел}

\section{Анализ спектральной оптической плотности плазмы}

\begin{figure}[ht]
	\noindent\begin{tikzpicture}
		\begin{axis}[
			xlabel = {$\nu$},
			ylabel = {$\tau$},
			xmin = 0, xmax = 2.8e+15,
			grid = both,
			width = \linewidth,
			height = 0.5\linewidth,
		]
			\addplot table [x=nu, y=tau]{inc/data/optical_density.dat};
		\end{axis}
	\end{tikzpicture}

	\noindent\begin{tikzpicture}
		\begin{axis}[
			xlabel = {$\nu$},
			ylabel = {$\tau$},
			domain=1e+14:4e+14,
			xmin = 1e+14, xmax = 7e+14,
			grid = both,
			width = 0.485\linewidth,
			height = 0.5\linewidth,
		]
			\addplot table [x=nu, y=tau]{inc/data/optical_density.dat};
	\end{axis}
	\end{tikzpicture}\begin{tikzpicture}
		\begin{semilogyaxis}[
			xlabel = {$\nu$},
			ylabel = {$\tau$},
			xmin = 19e+14, xmax = 28e+14,
			grid = both,
			width = 0.485\linewidth,
			height = 0.5\linewidth,
		]
			\addplot table [x=nu, y=tau]{inc/data/optical_density.dat};
		\end{semilogyaxis}
	\end{tikzpicture}
	\captionsetup{justification=centering}
	\caption{Спектральная оптическая плотность разрядной плазмы}
	\label{plt:optical-density}
\end{figure}

На рисунке \ref{plt:optical-density} представлена спектральная оптическая плотность разрядной плазмы радиусом $R = 0,35$ см, осевой и краевой температурами $T_0 = 10000$ К, $T_w = 2000$ К, показателем степени $m = 4$ (см. \eqref{eqn:xenon-temperature}) и зеркальным отражателем на поверхности плазмы с коэффициентом отражения $\rho = 0,95$, расчитываемая по формуле:
\begin{equation}
	\tau = \int\displaylimits_0^R k_{\text{погл}} \,\mathrm dr.
\end{equation}

Диапазоны частот, соответствующие критически малым значениям оптической плотности, являющиеся по сути прозрачными для светового излучения, не представляют интереса для анализа физико-оптических свойств разрядной плазмы так же, как и не представляют интерес диапазоны частот, соответствующие большим значениям оптической плотности, в которых плазма выступает в качестве чёрного газа. В дальнейшем расчёты поглощённой мощности по объёму будут осущевлятся в диапазонах $[0,36232\cdot10^{15}, 0,36240\cdot10^{15}]$ c $\tau_{81} = 0,691569$ и $[2,38460\cdot10^{15}, 2,38960\cdot10^{15}]$ с $\tau_{170} = 2,507276$.

На рисунке \ref{plt:rho} представлена зависимость распределения поглощённой мощности по объёму разрядной плазмы при переменном коэффициенте $\rho$ зеркального отражения на поверхности.

На рисунке \ref{plt:m} представлена зависимость распределения поглощённой мощности по объёму разрядной плазмы при переменном показателе степени $m$ убывания абсолютной температуры.

На рисунке \ref{plt:t} представлена зависимость распределения поглощённой мощности по объёму разрядной плазмы при переменном значении осевой абсолютной температуры $T_0$.

\begin{figure}[ht]
	\noindent\begin{tikzpicture}
		\begin{axis}[
			xlabel = {$z$},
			ylabel = {$I \;\; [\text{Вт}/\text{см}^2]$},
			xmin = 0, xmax = 1.025,
			grid = both,
			width = \linewidth,
			height = 0.5\linewidth,
		]
			\addplot table [x=zmax, y=mmb2]{inc/data/Xe_R=0.35.dat};
			\addlegendentry{$\rho = 0.90$}
			\addplot table [x=zmax, y=mmm2]{inc/data/Xe_R=0.35.dat};
			\addlegendentry{$\rho = 0.95$}
			\addplot table [x=zmax, y=mmt2]{inc/data/Xe_R=0.35.dat};
			\addlegendentry{$\rho = 0.98$}
		\end{axis}
	\end{tikzpicture}

	\noindent\begin{tikzpicture}
		\begin{axis}[
			xlabel = {$z$},
			ylabel = {$I \;\; [\text{Вт}/\text{см}^3]$},
			xmin = 0, xmax = 1.025,
			grid = both,
			width = 0.98\linewidth,
			height = 0.5\linewidth,
		]
			\addplot table [x=zmax, y=mmb3]{inc/data/Xe_R=0.35.dat};
			\addlegendentry{$\rho = 0.90$}
			\addplot table [x=zmax, y=mmm3]{inc/data/Xe_R=0.35.dat};
			\addlegendentry{$\rho = 0.95$}
			\addplot table [x=zmax, y=mmt3]{inc/data/Xe_R=0.35.dat};
			\addlegendentry{$\rho = 0.98$}
		\end{axis}
	\end{tikzpicture}
	\captionsetup{justification=centering}
	\caption{Поглощённая мощность плазмы, $\Delta\nu = [0,36232\cdot10^{15}, 0,36240\cdot10^{15}]$ $R = 0,35$ см, $T_0 = 10000$ К, $T_w = 2000$ К, $m = 4$}
	\label{plt:rho}
\end{figure}

\begin{figure}[ht]
	\noindent\begin{tikzpicture}
		\begin{axis}[
			xlabel = {$z$},
			ylabel = {$I \;\; [\text{Вт}/\text{см}^2]$},
			xmin = 0, xmax = 1.025,
			yticklabel style={/pgf/number format/.cd,fixed,precision=2},
			grid = both,
			width = 0.965\linewidth,
			height = 0.5\linewidth,
		]
			\addplot table [x=zmax, y=mbm2]{inc/data/Xe_R=0.35.dat};
			\addlegendentry{$m = 2$}
			\addplot table [x=zmax, y=mmm2]{inc/data/Xe_R=0.35.dat};
			\addlegendentry{$m = 4$}
			\addplot table [x=zmax, y=mtm2]{inc/data/Xe_R=0.35.dat};
			\addlegendentry{$m = 8$}
		\end{axis}
	\end{tikzpicture}

	\noindent\begin{tikzpicture}
		\begin{axis}[
			xlabel = {$z$},
			ylabel = {$I \;\; [\text{Вт}/\text{см}^3]$},
			xmin = 0, xmax = 1.025,
			grid = both,
			width = 0.98\linewidth,
			height = 0.5\linewidth,
		]
			\addplot table [x=zmax, y=mbm3]{inc/data/Xe_R=0.35.dat};
			\addlegendentry{$m = 2$}
			\addplot table [x=zmax, y=mmm3]{inc/data/Xe_R=0.35.dat};
			\addlegendentry{$m = 4$}
			\addplot table [x=zmax, y=mtm3]{inc/data/Xe_R=0.35.dat};
			\addlegendentry{$m = 8$}
		\end{axis}
	\end{tikzpicture}
	\captionsetup{justification=centering}
	\caption{Поглощённая мощность плазмы, $\Delta\nu = [0,36232\cdot10^{15}, 0,36240\cdot10^{15}]$ $R = 0,35$ см, $T_0 = 10000$ К, $T_w = 2000$ К, $rho = 0,95$}
	\label{plt:m}
\end{figure}

\begin{figure}[ht]
	\noindent\begin{tikzpicture}
		\begin{axis}[
			xlabel = {$z$},
			ylabel = {$I \;\; [\text{Вт}/\text{см}^2]$},
			xmin = 0, xmax = 1.025,
			yticklabel style={/pgf/number format/.cd,fixed,precision=2},
			grid = both,
			width = 0.965\linewidth,
			height = 0.5\linewidth,
		]
			\addplot table [x=zmax, y=bmm2]{inc/data/Xe_R=0.35.dat};
			\addlegendentry{$T_0 = 8000 \; \text{К}$}
			\addplot table [x=zmax, y=mmm2]{inc/data/Xe_R=0.35.dat};
			\addlegendentry{$T_0 = 10000 \; \text{К}$}
			\addplot table [x=zmax, y=tmm2]{inc/data/Xe_R=0.35.dat};
			\addlegendentry{$T_0 = 12000 \; \text{К}$}
		\end{axis}
	\end{tikzpicture}

	\noindent\begin{tikzpicture}
		\begin{axis}[
			xlabel = {$z$},
			ylabel = {$I \;\; [\text{Вт}/\text{см}^3]$},
			xmin = 0, xmax = 1.025,
			grid = both,
			width = 0.98\linewidth,
			height = 0.5\linewidth,
		]
			\addplot table [x=zmax, y=bmm3]{inc/data/Xe_R=0.35.dat};
			\addlegendentry{$T_0 = 8000 \; \text{К}$}
			\addplot table [x=zmax, y=mmm3]{inc/data/Xe_R=0.35.dat};
			\addlegendentry{$T_0 = 10000 \; \text{К}$}
			\addplot table [x=zmax, y=tmm3]{inc/data/Xe_R=0.35.dat};
			\addlegendentry{$T_0 = 12000 \; \text{К}$}
		\end{axis}
	\end{tikzpicture}
	\captionsetup{justification=centering}
	\caption{Поглощённая мощность плазмы, $\Delta\nu = [0,36232\cdot10^{15}, 0,36240\cdot10^{15}]$ $R = 0,35$ см, $T_w = 2000$ К, $m = 4$, $rho = 0,95$}
	\label{plt:t}
\end{figure}

\section*{Выводы}
