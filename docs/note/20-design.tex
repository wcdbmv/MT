%% Методические указания к выполнению, оформлению и защите выпускной квалификационной работы бакалавра
%% 2.5 Конструкторский раздел
%%
%% В конструкторском разделе описывается разрабатываемый и/или модифицируемый метод или алгоритм.
%%
%% В случае если в бакалаврском проекте разрабатывается новый метод или алгоритм, необходимо подробно изложить их суть, привести все необходимые для их реализации математические выкладки, обосновать последовательность этапов выполнения.
%% При этом для каждого этапа следует выделить необходимые исходные данные и получаемые результаты.
%%
%% При использовании известного алгоритма следует указать специфические особенности его практической реализации, присущие решаемой задаче, и пути их решения в ходе программирования.
%% Для описания метода или алгоритма необходимо выбрать наиболее подходящую форму записи (схема (ГОСТ 19.701-90), диаграмма деятельности, псевдокод и т. п.).
%% Учитывая, что на эффективность алгоритма непосредственно влияют используемые структуры данных, в данном разделе РПЗ целесообразно провести сравнительный анализ структур, которые могут быть применены в рамках программной реализации выбранного алгоритма, и обосновать выбор одной из них.
%% В конце описания разработанного и/или модифицируемого алгоритма должны быть приведены выбранные способы тестирования и сами тесты.
%%
%% Перед формированием тестовых наборов данных целесообразно указать выделенные классы эквивалентности.
%% В данной части расчетно-пояснительной записки могут также выполняться расчеты для определения объемов памяти, необходимой для хранения данных, промежуточных и окончательных результатов работы программы, а также расчеты, позволяющие оценить время решения задачи на ЭВМ.
%% Эти результаты могут использоваться для обоснования правильности выбора метода и/или алгоритма из имеющихся альтернативных вариантов, а также для оценки возможности практически реализовать поставленную задачу на имеющейся технической базе.
%%
%% Другой важный момент, который должен найти свое отражение в конструкторском разделе, это описание структуры разрабатываемого программного обеспечения.
%% Обычно оно включает в себя:
%% — описание общей структуры — определение основных частей (компонентов) и их взаимосвязей по управлению и по данным;
%% — декомпозицию компонентов и построение структурных иерархий;
%% — проектирование компонентов.
%%
%% Для графического представления такого описания, если есть необходимость, следует использовать:
%% — функциональную модель IDEF0 с декомпозицией решения исходной задачи на несколько уровней (разрабатываемые модули обычно играют роль механизмов);
%% — спецификации компонентов (процессов);
%% — модель данных (ER-диаграмма);
%% — диаграмму классов;
%% — диаграмму компонентов;
%% — диаграмму переходов состояний (конечный автомат), характеризующих поведение системы во времени.
%%
%% Рекомендуемый объем конструкторского раздела 25—30 страниц.

\chapter{Конструкторский раздел}

\section{Формирование объёмного излучения}

Для трассировки траектории фотона необходимо помимо начального положения определить и его направление.
Световое излучение представляет собой сферическую волну, поэтому генерация направлений распространения фотонов аналогична задаче генерации точек на поверхности сферы.

Задача равномерного распределения точек на сфере имеет очень долгую историю.
Она обладает огромной важностью во многих областях математики, физики, химии и вычислений, в том числе в численном анализе, теории аппроксимации, кристаллографии, морфологии вирусов, электростатике, теории кодирования и компьютерной графике.

К сожалению, она так и не была точно решена, за исключением небольшого количества частных случаев.
Следовательно, практически во всех ситуациях мы можем только надеяться найти близкие к оптимальному решения этой задачи.

Из всех этих почти оптимальных решений один из самых простых способов основан на решётке Фибоначчи, или на золотой спирали.
Более того, в отличие от большинства других итеративных или рандомизированных способов, например, имитации отжига, спираль Фибоначчи является одним из немногих способов непосредственного построения, работающих для произвольного количества точек $n$.

Стандартное современное определение решётки Фибоначчи, равномерно распределяющей $n$ точек внутри единичного квадрата $[0, 1)^2$, имеет вид:

\begin{equation}
	t_i = (x_i, y_i) = \left(\left\{\frac{i}{\varphi}\right\}, \frac{i}{n}\right) \text{ для } 0 \leqslant i < n,
\end{equation}

\noindent где $\displaystyle \varphi = \frac{1+\sqrt5}{2}$, а оператор фигурные скобки обозначает дробную часть аргумента.

Эти множества точек можно наложить с единичного квадрата $[0, 1)^2$ на сферу $S^2$ при помощи цилиндрического равноплощадного проецирования:
\begin{align}
	(x, y) \to (\theta, \phi)&\colon (2 \pi x, \arccos{(1 - 2y)}, \\
	(\theta, \phi) \to (x, y, z)&\colon (\cos\theta \sin\phi, \sin\theta \sin\phi, \cos\phi).
\end{align}

Стоит заметить, что существует множество стандартов обозначений углов в сферической геометрии.
В приведённой выше формуле используется нотация, в которой $\theta \in [0, 2\pi]$ является долготой, а $\phi \in [0, \pi]$ — углом от условного севера.

То есть, решётка Фибоначчи — это простой способ очень равномерного распределения точек в прямоугольнике, на диске или поверхности сферы.
Преимущество такого способа генерации светового излучения состоит в том, что по каждому направлению объёмная интенсивность будет примерно одинаковая.
Другой подход, более точный, заключается в генерации каркасной сетки на сфере по меридианам и широтам:

\begin{equation}
	\begin{cases}
		x = \sin{\left(\pi \cdot \frac mM\right)}\cos{\left(2\pi \cdot \frac nN\right)}, \\
		y = \sin{\left(\pi \cdot \frac mM\right)}\sin{\left(2\pi \cdot \frac nN\right)}, \\
		z = \cos{\left(\pi \cdot \frac mM\right)}, \\
		m = \{0, \ldots, M\}, n = \{0, \ldots, N-1\},
	\end{cases}
\end{equation}

\noindent где $M$ — количество широт, $N$ — меридианов.

При таком способе объёмная интенсивность фотонов не равномерна и зависит от переменного телесного угла:


\begin{equation}
	\mathrm{d}\Omega = \sin{\theta} \, \mathrm{d}\theta \, \mathrm{d}\phi,
\end{equation}

\noindent где $\displaystyle \theta = \frac{\pi}{M}$, $\displaystyle \phi = \frac{2\pi}{N}$.

На рисунке \ref{img:fibonacci-and-wire-spheres} — визуализация двух способов генерации точек на сфере.

\begin{figure}[ht]
	\center{
		\includegraphics[height=80mm]{inc/img/fibonacci-sphere}
		\includegraphics[height=80mm]{inc/img/wire-sphere}
	}
	\captionsetup{justification=centering}
	\caption{Сетка Фибоначчи и каркасная сетка}
	\label{img:fibonacci-and-wire-spheres}
\end{figure}

\section{Моделирование траектории луча}

Луч — это часть прямой, состоящая из заданной точки (начала) и всех точек, лежащих по определённую сторону от неё.
В настоящей работе удобнее задавать луч вектором начала $\vec P = (P_x, P_y, P_z)$ и направления $\vec D = (D_x, D_y, D_z)$.
В таком случае уравнение луча в трёхмерном пространстве имеет параметрический вид:

\begin{equation}
	\vec R = \vec P + \vec D \cdot t, \; t \geqslant 0.
\end{equation}

При достижении границы раздела двух сред прямолинейная траектория луча меняется, часть луча отражается, часть — преломляется.

Результирующий луч согласно закону зеркального отражения рассчитывается по формуле:

\begin{equation}
	\label{eqn:reflection}
	\vec R = \vec I = 2\left(\vec I \cdot \vec N\right)\vec N,
\end{equation}

\noindent где $\vec I$ — луч падающий, $\vec N$ — нормаль к поверхности в точке падения.

Закон Снеллиуса в векторной форме задаёт координаты преломлённого луча:

\begin{equation}
	\label{eqn:refraction-begin}
	\mu = \frac{\eta_i}{\eta_t},
\end{equation}
\begin{equation}
	\label{eqn:snells-law-g}
	g = \sqrt{1 - \mu^2\left(1 - \left(\vec I \cdot \vec N\right)\right)^2},
\end{equation}
\begin{equation}
	\vec T = g \vec N + \mu \left(\vec I - \left(\vec I \cdot \vec N\right) \vec N\right),
\end{equation}

\noindent где $\eta_i$, $\eta_t$ — показатели преломления сред падающего $\vec I$ и преломлённого $\vec T$ лучей соответственно.

Если подкоренное выражение у переменной $g$ \eqref{eqn:snells-law-g} меньше нуля, то происходит полное внутреннее отражение.

Доли энергий отражённого и преломлённого лучей соответственно рассчитывается по формулам Френеля:

\begin{equation}
	R = \frac12 \left(\frac{g-c}{g+c}\right)^2 \left(1 + \left(\frac{c(g + c) - \mu^2}{c(g - c) + \mu^2}\right)^2\right)
\end{equation}
\begin{equation}
	\label{eqn:refraction-end}
	T = 1 - R.
\end{equation}

\section{Геометрическое описание моделируемых сред}

По озвученным в аналитическом разделе ограничениям в качестве объектов в моделируемой системе рассматриваются цилиндрические симметрии: бесконечные (по оси Z) цилиндр и эллиптический цилиндр.
Цилиндр задаётся центром $\vec C$ и радиусом $R$, эллиптический цилиндр~— центром $\vec C$, длиной большой оси $2a$ и малой — $2b$.
Формы оболочек таких сред описывают распространённые до трёхмерного пространства уравнения окружности и эллипса:

\begin{equation}
	\label{eqn:cylinder}
	(x - C_x)^2 + (y - C_y)^2 = R^2,
\end{equation}
\begin{equation}
	\frac{(x - C_x)^2}{a^2} + \frac{(y - C_y)^2}{b^2} = 1.
\end{equation}

Моделирование неоднородности по объёму распределения оптических свойств и температуры осуществляется за счёт рассмотрения областей между оболочками близлежащих концентрических (с задаваемым шагом $h$) цилиндрических симметрий.

Необходимые для расчётов отражения и преломления лучей света уравнения перпендикуляров для цилиндра и эллиптического цилиндра соответственно:

\begin{equation}
	\overrightarrow{P_{\text{ц}}} = (S_x - C_x, S_y - C_y, 0),
\end{equation}
\begin{equation}
	\overrightarrow{P_{\text{эц}}} = \left(\frac{S_x - C_x}{a^2}, \frac{S_y - C_y}{b^2}, 0\right),
\end{equation}

где $\vec S = \left(S_x, S_y, S_z\right)$ — точка на поверхности оболочки.

Для вывода формул пересечения луча с поверхностью среды необходимо задать луч в параметрическом виде:

\begin{equation}
	\label{eqn:ray-parametric}
	\begin{cases}
		x = P_x + D_x t, \\
		y = P_y + D_y t, \\
		z = P_z + D_z t, \\
		t \geqslant 0,
	\end{cases}
\end{equation}

где $\vec P = (P_x, P_y, P_z)$ — начало луча, $\vec D = (D_x, D_y, D_z)$ — нормализованное невырожденное направление луча.

Опишем вывод формулы точек пересечения луча с поверхностью бесконечного по оси Z цилиндра. Точки пересечения, если таковые имеются, удовлетворяют уравнениям луча \eqref{eqn:ray-parametric} и цилиндра \eqref{eqn:cylinder}:

\begin{equation}
	\label{eqn:cylinder-solve-begin}
	(P_x + D_x t - C_x)^2 + (P_y + D_y t - C_y)^2 = R^2.
\end{equation}

\noindent Положим $u_x = P_x - C_x$, $u_y = P_y - C_y$, тогда:

\begin{equation}
	(D_x t + u_x)^2 + (D_y t + u_y)^2 = R^2 \Leftrightarrow
\end{equation}
\begin{equation}
	D_x^2 t^2 + 2D_x u_x t + u_x^2 + D_y^2 t^2 + 2D_y u_y t + u_y^2 - R^2 = 0 \Leftrightarrow
\end{equation}
\begin{equation}
	\label{eqn:cylinder-quadratic-equation}
	\left(D_x^2 + D_y^2\right) t^2 + 2(D_x u_x + D_y u_y)t + u_x^2 + u_y^2 - R^2 = 0.
\end{equation}

Если четверть дискриминанта $\displaystyle \frac D4 = (D_x u_x + D_y u_y)^2 - \left(D_x^2 + D_y^2\right)\left(u_x^2 + u_y^2-R^2\right)$ меньше нуля, то точек пересечения нет.
Если луч направлен по оси Z (то есть $D_x^2 + D_y^2 = 0$), то он либо полностью лежит на поверхности цилиндра, либо её не пересекает.
Такие случаи не представляют интерес для исследования, так как в конечном итоге описанные лучи полностью поглотятся в среде без необходимости рассчитывать их дальнейшие траектории.
Решая квадратное уравнение \eqref{eqn:cylinder-quadratic-equation}, получаем возможные значения параметра $t$ уравнения луча \eqref{eqn:ray-parametric}:

\begin{equation}
	\label{eqn:cylinder-possible-t}
	t = \frac{-D_x u_x - D_y u_y \pm \sqrt{(D_x u_x + D_y u_y)^2 - \left(D_x^2 + D_y^2\right)\left(u_x^2 + u_y^2 - R^2\right)}}{D_x^2+D_y^2}.
\end{equation}

Среди возможных значений параметра $t$ \eqref{eqn:cylinder-possible-t} необходимо выбрать наименьшее неотрицательное, обозначим его, при наличии, за $t^+$.
Таким образом, формула точки пересечения луча с цилиндрической поверхностью имеет вид:

\begin{equation}
	\vec I = \vec P + \vec D \cdot t^+.
\end{equation}

Аналогично \eqref{eqn:cylinder-solve-begin} — \eqref{eqn:cylinder-quadratic-equation}, получим квадратное уравнение пересечения луча и бесконечного по оси Z эллиптического цилиндра:

\begin{equation}
	\frac{(P_x + D_x t - C_x)^2}{a^2} + \frac{(P_y + D_y t - C_y)^2}{b^2} = 1 \Leftrightarrow
\end{equation}
\begin{equation}
	b^2(P_x + D_x t - C_x)^2 + a^2(P_y + D_y t - C_y)^2 = a^2 b^2 \Leftrightarrow
\end{equation}
\begin{equation}
	b^2(D_x t + u_x)^2 + a^2(D_y t + u_y)^2 = a^2b^2 \Leftrightarrow
\end{equation}
\begin{equation}
	b^2\left(D_x^2 t^2 + 2D_x u_x t + u_x^2\right) + a^2\left(D_y^2 t^2 + 2D_y u_y t + u_y^2\right) - a^2b^2 = 0 \Leftrightarrow
\end{equation}
\begin{equation}
	\left(b^2 D_x^2 + a^2 D_y^2\right)t^2 + 2\left(b^2 D_x u_x + a^2 D_y u_y\right)t + b^2 u_x^2 + a^2 u_y^2 - a^2b^2 = 0.
\end{equation}

Отсюда

\begin{equation}
	\label{eqn:elliptic-cylinder-possible-t}
	\begin{matrix}
		t = \frac{1}{b^2 D_x^2 + a^2 D_y^2}\bigg[-b^2 D_x u_x - a^2 D_y u_y \pm \\
		\sqrt{\left(b^2 D_x u_x + a^2 D_y u_y\right)^2 - \left(b^2 D_x^2 + a^2 D_y^2\right)\left(b^2 u_x^2 + a^2 u_y^2 - a^2 b^2\right)}\bigg].
	\end{matrix}
\end{equation}

\section{Моделирование физических свойств неоднородных сред}

Закон излучения Планка описывает спектральное распределение энергии электромагнитного излучения, находящегося в тепловом равновесии с веществом при заданной температуре.
Идеализированной моделью равновесного излучения служит электромагнитное поле внутри полости, расположенной в нагретом веществе при условии, что стенки вещества непрозрачны для излучения.
Спектр такого равновесного излучения называют спектром излучения абсолютно чёрного тела.

\begin{equation}
	u_\nu(T, \nu) = \frac{8\pi h\nu^3}{c^3\exp{\left(\frac{h\nu}{kT} - 1\right)}} \; \left[\text{Дж}\cdot\text{с}/\text{м}^3\right],
\end{equation}

\noindent где $h = 6,62607015 \cdot 10^{-34} \; \left[\text{Дж}\cdot\text{с}\right]$~— постоянная Планка, $c = 299792458 \; \left[\text{м}/\text{с}\right]$~— скорость света в вакууме, $k = 1,380649 \cdot 10^{-23} \; \left[\text{Дж}/\text{К}\right]$~— постоянная Больцмана, $\nu \; \left[\text{Дж}\right]$~— частота излучения, $T \; \left[\text{К}\right]$~— абсолютная температура.

Интенсивность электромагнитного излучения рассчитывается по формуле

\begin{equation}
	I = \frac{u_\nu(T, \nu)c\Delta\nu}{4\pi} \; \left[\text{Вт}/\text{м}^2\right],
\end{equation}

где $\Delta\nu \; \left[\text{Дж}\right]$ — ширина диапазона частоты излучения.

Среда распространения фотона, если это не вакуум, непрерывно поглощает его интенсивность.
Расчёт новой интенсивность фотона, прошедшего малый участок $\Delta r$ пути в участке среды с коэффициентом оптического поглощения $k_{\text{погл}}$:

\begin{equation}
	\label{eqn:intensity-begin}
	I^* = Ie^{-k_{\text{погл}\Delta r}},
\end{equation}

\noindent и поглощённая интенсивность:

\begin{equation}
	\label{eqn:intensity-end}
	\Delta I = I - I^*.
\end{equation}

Суть метода настоящей работы в том числе выражается в учёте неоднородности физико-оптических свойств сред, а именно температуры и коэффициента поглощения.

Обе характеристики могут задаваться формулами.
Например, распределение температуры плазмы может быть описано, как

\begin{equation}
	T = T_0 + (T_w - T_0)z^m,
\end{equation}

где $z$ — безразмерный радиус, $T_0$ и $T_w$ — температуры при $z = 0$ и $z = 1$ соответственно, $m$ — показатель степени в диапазоне 2–8.
Как видно из рисунка \ref{img:xenon-temperature}, такое распределение достаточно близко описывает то, что происходит в действительности.

\img{width=\linewidth}{xenon-temperature}{Радиальные температурные распределения в разряде Cs‑Hg‑Xe. p=0.1 МПа}

Пример аппроксимации неоднородности коэффициентов поглощения плазмы и кварца соответственно:

\begin{equation}
	k_{\text{пл}} = 0,04 \cdot \left(\frac{T}{2000}\right)^2,
\end{equation}
\begin{equation}
	k_{\text{кв}} = 0,001 \cdot \left(\frac{T}{300}\right)^{1,5}.
\end{equation}

Для плазмы коэффициент оптического поглощения может быть рассчитан из таблицы (часть таблицы см. в приложении А, стр. \pageref{toc:attachment-a}).
Интерполяцию значений лучше производить в логарифмических координатах:

\begin{equation}
	\begin{matrix}
		k_{\text{пл}}(T, \nu) = \\
		\exp{\left(\ln{k_t(T_{min}, \nu)} + \frac{(\ln{T} - \ln{T_{min}})(\ln{k_t(T_{max}, \nu)} - \ln{k_t(T_{min}, \nu)})}{\ln{T_{max}} - \ln{T_{min}}}\right)},
	\end{matrix}
\end{equation}

\noindent где $T_{min}$, $T_{max}$ — ближайший табличный диапазон температур, $k_t$~— табличное значение коэффициента оптического поглощения плазмы.
На рисунке \ref{img:xenon-absorption-coefficient} представлено распределение показателя поглощения при температуре 3500 К.

\img{width=\linewidth}{xenon-absorption-coefficient}{Коэффициент оптического поглощения плазмы Cs-Hg-Xe. Давление p=0.1 МПа, температура T=3500 K}

\section{Дискретно-лучевой метод моделирования световых полей в системах сложной конфигурации}

На рисунке \ref{img:A1} представлена детализированная диаграмма IDEF0 компонента A0.

\img{width=\linewidth}{A1}{Диаграмма IDEF0 компонента A0}

На рисунке \ref{img:worker} представлена схема алгоритма параллельной трассировки лучей.

\img{width=\linewidth}{worker}{Схема алгоритма параллельной трассировки лучей}

На рисунке \ref{img:ray-tracing} представлена схема алгоритм трассировки луча:
\begin{itemize}
	\item поиск ближайшей поверхности пересечения — поиск поверхности с наименьшим значением $t^+$: \eqref{eqn:cylinder-possible-t}, \eqref{eqn:elliptic-cylinder-possible-t};
	\item геометрическое отражение: \eqref{eqn:reflection};
	\item преломление: \eqref{eqn:refraction-begin} — \eqref{eqn:refraction-end};
	\item перерасчёт интенсивности: \eqref{eqn:intensity-begin} — \eqref{eqn:intensity-end}.
\end{itemize}

\img{width=\linewidth}{ray-tracing}{Схема алгоритма трассировки луча}
